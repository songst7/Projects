
% Default to the notebook output style

    


% Inherit from the specified cell style.




    
\documentclass[11pt]{article}

    
    
    \usepackage[T1]{fontenc}
    % Nicer default font (+ math font) than Computer Modern for most use cases
    \usepackage{mathpazo}

    % Basic figure setup, for now with no caption control since it's done
    % automatically by Pandoc (which extracts ![](path) syntax from Markdown).
    \usepackage{graphicx}
    % We will generate all images so they have a width \maxwidth. This means
    % that they will get their normal width if they fit onto the page, but
    % are scaled down if they would overflow the margins.
    \makeatletter
    \def\maxwidth{\ifdim\Gin@nat@width>\linewidth\linewidth
    \else\Gin@nat@width\fi}
    \makeatother
    \let\Oldincludegraphics\includegraphics
    % Set max figure width to be 80% of text width, for now hardcoded.
    \renewcommand{\includegraphics}[1]{\Oldincludegraphics[width=.8\maxwidth]{#1}}
    % Ensure that by default, figures have no caption (until we provide a
    % proper Figure object with a Caption API and a way to capture that
    % in the conversion process - todo).
    \usepackage{caption}
    \DeclareCaptionLabelFormat{nolabel}{}
    \captionsetup{labelformat=nolabel}

    \usepackage{adjustbox} % Used to constrain images to a maximum size 
    \usepackage{xcolor} % Allow colors to be defined
    \usepackage{enumerate} % Needed for markdown enumerations to work
    \usepackage{geometry} % Used to adjust the document margins
    \usepackage{amsmath} % Equations
    \usepackage{amssymb} % Equations
    \usepackage{textcomp} % defines textquotesingle
    % Hack from http://tex.stackexchange.com/a/47451/13684:
    \AtBeginDocument{%
        \def\PYZsq{\textquotesingle}% Upright quotes in Pygmentized code
    }
    \usepackage{upquote} % Upright quotes for verbatim code
    \usepackage{eurosym} % defines \euro
    \usepackage[mathletters]{ucs} % Extended unicode (utf-8) support
    \usepackage[utf8x]{inputenc} % Allow utf-8 characters in the tex document
    \usepackage{fancyvrb} % verbatim replacement that allows latex
    \usepackage{grffile} % extends the file name processing of package graphics 
                         % to support a larger range 
    % The hyperref package gives us a pdf with properly built
    % internal navigation ('pdf bookmarks' for the table of contents,
    % internal cross-reference links, web links for URLs, etc.)
    \usepackage{hyperref}
    \usepackage{longtable} % longtable support required by pandoc >1.10
    \usepackage{booktabs}  % table support for pandoc > 1.12.2
    \usepackage[inline]{enumitem} % IRkernel/repr support (it uses the enumerate* environment)
    \usepackage[normalem]{ulem} % ulem is needed to support strikethroughs (\sout)
                                % normalem makes italics be italics, not underlines
    

    
    
    % Colors for the hyperref package
    \definecolor{urlcolor}{rgb}{0,.145,.698}
    \definecolor{linkcolor}{rgb}{.71,0.21,0.01}
    \definecolor{citecolor}{rgb}{.12,.54,.11}

    % ANSI colors
    \definecolor{ansi-black}{HTML}{3E424D}
    \definecolor{ansi-black-intense}{HTML}{282C36}
    \definecolor{ansi-red}{HTML}{E75C58}
    \definecolor{ansi-red-intense}{HTML}{B22B31}
    \definecolor{ansi-green}{HTML}{00A250}
    \definecolor{ansi-green-intense}{HTML}{007427}
    \definecolor{ansi-yellow}{HTML}{DDB62B}
    \definecolor{ansi-yellow-intense}{HTML}{B27D12}
    \definecolor{ansi-blue}{HTML}{208FFB}
    \definecolor{ansi-blue-intense}{HTML}{0065CA}
    \definecolor{ansi-magenta}{HTML}{D160C4}
    \definecolor{ansi-magenta-intense}{HTML}{A03196}
    \definecolor{ansi-cyan}{HTML}{60C6C8}
    \definecolor{ansi-cyan-intense}{HTML}{258F8F}
    \definecolor{ansi-white}{HTML}{C5C1B4}
    \definecolor{ansi-white-intense}{HTML}{A1A6B2}

    % commands and environments needed by pandoc snippets
    % extracted from the output of `pandoc -s`
    \providecommand{\tightlist}{%
      \setlength{\itemsep}{0pt}\setlength{\parskip}{0pt}}
    \DefineVerbatimEnvironment{Highlighting}{Verbatim}{commandchars=\\\{\}}
    % Add ',fontsize=\small' for more characters per line
    \newenvironment{Shaded}{}{}
    \newcommand{\KeywordTok}[1]{\textcolor[rgb]{0.00,0.44,0.13}{\textbf{{#1}}}}
    \newcommand{\DataTypeTok}[1]{\textcolor[rgb]{0.56,0.13,0.00}{{#1}}}
    \newcommand{\DecValTok}[1]{\textcolor[rgb]{0.25,0.63,0.44}{{#1}}}
    \newcommand{\BaseNTok}[1]{\textcolor[rgb]{0.25,0.63,0.44}{{#1}}}
    \newcommand{\FloatTok}[1]{\textcolor[rgb]{0.25,0.63,0.44}{{#1}}}
    \newcommand{\CharTok}[1]{\textcolor[rgb]{0.25,0.44,0.63}{{#1}}}
    \newcommand{\StringTok}[1]{\textcolor[rgb]{0.25,0.44,0.63}{{#1}}}
    \newcommand{\CommentTok}[1]{\textcolor[rgb]{0.38,0.63,0.69}{\textit{{#1}}}}
    \newcommand{\OtherTok}[1]{\textcolor[rgb]{0.00,0.44,0.13}{{#1}}}
    \newcommand{\AlertTok}[1]{\textcolor[rgb]{1.00,0.00,0.00}{\textbf{{#1}}}}
    \newcommand{\FunctionTok}[1]{\textcolor[rgb]{0.02,0.16,0.49}{{#1}}}
    \newcommand{\RegionMarkerTok}[1]{{#1}}
    \newcommand{\ErrorTok}[1]{\textcolor[rgb]{1.00,0.00,0.00}{\textbf{{#1}}}}
    \newcommand{\NormalTok}[1]{{#1}}
    
    % Additional commands for more recent versions of Pandoc
    \newcommand{\ConstantTok}[1]{\textcolor[rgb]{0.53,0.00,0.00}{{#1}}}
    \newcommand{\SpecialCharTok}[1]{\textcolor[rgb]{0.25,0.44,0.63}{{#1}}}
    \newcommand{\VerbatimStringTok}[1]{\textcolor[rgb]{0.25,0.44,0.63}{{#1}}}
    \newcommand{\SpecialStringTok}[1]{\textcolor[rgb]{0.73,0.40,0.53}{{#1}}}
    \newcommand{\ImportTok}[1]{{#1}}
    \newcommand{\DocumentationTok}[1]{\textcolor[rgb]{0.73,0.13,0.13}{\textit{{#1}}}}
    \newcommand{\AnnotationTok}[1]{\textcolor[rgb]{0.38,0.63,0.69}{\textbf{\textit{{#1}}}}}
    \newcommand{\CommentVarTok}[1]{\textcolor[rgb]{0.38,0.63,0.69}{\textbf{\textit{{#1}}}}}
    \newcommand{\VariableTok}[1]{\textcolor[rgb]{0.10,0.09,0.49}{{#1}}}
    \newcommand{\ControlFlowTok}[1]{\textcolor[rgb]{0.00,0.44,0.13}{\textbf{{#1}}}}
    \newcommand{\OperatorTok}[1]{\textcolor[rgb]{0.40,0.40,0.40}{{#1}}}
    \newcommand{\BuiltInTok}[1]{{#1}}
    \newcommand{\ExtensionTok}[1]{{#1}}
    \newcommand{\PreprocessorTok}[1]{\textcolor[rgb]{0.74,0.48,0.00}{{#1}}}
    \newcommand{\AttributeTok}[1]{\textcolor[rgb]{0.49,0.56,0.16}{{#1}}}
    \newcommand{\InformationTok}[1]{\textcolor[rgb]{0.38,0.63,0.69}{\textbf{\textit{{#1}}}}}
    \newcommand{\WarningTok}[1]{\textcolor[rgb]{0.38,0.63,0.69}{\textbf{\textit{{#1}}}}}
    
    
    % Define a nice break command that doesn't care if a line doesn't already
    % exist.
    \def\br{\hspace*{\fill} \\* }
    % Math Jax compatability definitions
    \def\gt{>}
    \def\lt{<}
    % Document parameters
    \title{notebook}
    
    
    

    % Pygments definitions
    
\makeatletter
\def\PY@reset{\let\PY@it=\relax \let\PY@bf=\relax%
    \let\PY@ul=\relax \let\PY@tc=\relax%
    \let\PY@bc=\relax \let\PY@ff=\relax}
\def\PY@tok#1{\csname PY@tok@#1\endcsname}
\def\PY@toks#1+{\ifx\relax#1\empty\else%
    \PY@tok{#1}\expandafter\PY@toks\fi}
\def\PY@do#1{\PY@bc{\PY@tc{\PY@ul{%
    \PY@it{\PY@bf{\PY@ff{#1}}}}}}}
\def\PY#1#2{\PY@reset\PY@toks#1+\relax+\PY@do{#2}}

\expandafter\def\csname PY@tok@w\endcsname{\def\PY@tc##1{\textcolor[rgb]{0.73,0.73,0.73}{##1}}}
\expandafter\def\csname PY@tok@c\endcsname{\let\PY@it=\textit\def\PY@tc##1{\textcolor[rgb]{0.25,0.50,0.50}{##1}}}
\expandafter\def\csname PY@tok@cp\endcsname{\def\PY@tc##1{\textcolor[rgb]{0.74,0.48,0.00}{##1}}}
\expandafter\def\csname PY@tok@k\endcsname{\let\PY@bf=\textbf\def\PY@tc##1{\textcolor[rgb]{0.00,0.50,0.00}{##1}}}
\expandafter\def\csname PY@tok@kp\endcsname{\def\PY@tc##1{\textcolor[rgb]{0.00,0.50,0.00}{##1}}}
\expandafter\def\csname PY@tok@kt\endcsname{\def\PY@tc##1{\textcolor[rgb]{0.69,0.00,0.25}{##1}}}
\expandafter\def\csname PY@tok@o\endcsname{\def\PY@tc##1{\textcolor[rgb]{0.40,0.40,0.40}{##1}}}
\expandafter\def\csname PY@tok@ow\endcsname{\let\PY@bf=\textbf\def\PY@tc##1{\textcolor[rgb]{0.67,0.13,1.00}{##1}}}
\expandafter\def\csname PY@tok@nb\endcsname{\def\PY@tc##1{\textcolor[rgb]{0.00,0.50,0.00}{##1}}}
\expandafter\def\csname PY@tok@nf\endcsname{\def\PY@tc##1{\textcolor[rgb]{0.00,0.00,1.00}{##1}}}
\expandafter\def\csname PY@tok@nc\endcsname{\let\PY@bf=\textbf\def\PY@tc##1{\textcolor[rgb]{0.00,0.00,1.00}{##1}}}
\expandafter\def\csname PY@tok@nn\endcsname{\let\PY@bf=\textbf\def\PY@tc##1{\textcolor[rgb]{0.00,0.00,1.00}{##1}}}
\expandafter\def\csname PY@tok@ne\endcsname{\let\PY@bf=\textbf\def\PY@tc##1{\textcolor[rgb]{0.82,0.25,0.23}{##1}}}
\expandafter\def\csname PY@tok@nv\endcsname{\def\PY@tc##1{\textcolor[rgb]{0.10,0.09,0.49}{##1}}}
\expandafter\def\csname PY@tok@no\endcsname{\def\PY@tc##1{\textcolor[rgb]{0.53,0.00,0.00}{##1}}}
\expandafter\def\csname PY@tok@nl\endcsname{\def\PY@tc##1{\textcolor[rgb]{0.63,0.63,0.00}{##1}}}
\expandafter\def\csname PY@tok@ni\endcsname{\let\PY@bf=\textbf\def\PY@tc##1{\textcolor[rgb]{0.60,0.60,0.60}{##1}}}
\expandafter\def\csname PY@tok@na\endcsname{\def\PY@tc##1{\textcolor[rgb]{0.49,0.56,0.16}{##1}}}
\expandafter\def\csname PY@tok@nt\endcsname{\let\PY@bf=\textbf\def\PY@tc##1{\textcolor[rgb]{0.00,0.50,0.00}{##1}}}
\expandafter\def\csname PY@tok@nd\endcsname{\def\PY@tc##1{\textcolor[rgb]{0.67,0.13,1.00}{##1}}}
\expandafter\def\csname PY@tok@s\endcsname{\def\PY@tc##1{\textcolor[rgb]{0.73,0.13,0.13}{##1}}}
\expandafter\def\csname PY@tok@sd\endcsname{\let\PY@it=\textit\def\PY@tc##1{\textcolor[rgb]{0.73,0.13,0.13}{##1}}}
\expandafter\def\csname PY@tok@si\endcsname{\let\PY@bf=\textbf\def\PY@tc##1{\textcolor[rgb]{0.73,0.40,0.53}{##1}}}
\expandafter\def\csname PY@tok@se\endcsname{\let\PY@bf=\textbf\def\PY@tc##1{\textcolor[rgb]{0.73,0.40,0.13}{##1}}}
\expandafter\def\csname PY@tok@sr\endcsname{\def\PY@tc##1{\textcolor[rgb]{0.73,0.40,0.53}{##1}}}
\expandafter\def\csname PY@tok@ss\endcsname{\def\PY@tc##1{\textcolor[rgb]{0.10,0.09,0.49}{##1}}}
\expandafter\def\csname PY@tok@sx\endcsname{\def\PY@tc##1{\textcolor[rgb]{0.00,0.50,0.00}{##1}}}
\expandafter\def\csname PY@tok@m\endcsname{\def\PY@tc##1{\textcolor[rgb]{0.40,0.40,0.40}{##1}}}
\expandafter\def\csname PY@tok@gh\endcsname{\let\PY@bf=\textbf\def\PY@tc##1{\textcolor[rgb]{0.00,0.00,0.50}{##1}}}
\expandafter\def\csname PY@tok@gu\endcsname{\let\PY@bf=\textbf\def\PY@tc##1{\textcolor[rgb]{0.50,0.00,0.50}{##1}}}
\expandafter\def\csname PY@tok@gd\endcsname{\def\PY@tc##1{\textcolor[rgb]{0.63,0.00,0.00}{##1}}}
\expandafter\def\csname PY@tok@gi\endcsname{\def\PY@tc##1{\textcolor[rgb]{0.00,0.63,0.00}{##1}}}
\expandafter\def\csname PY@tok@gr\endcsname{\def\PY@tc##1{\textcolor[rgb]{1.00,0.00,0.00}{##1}}}
\expandafter\def\csname PY@tok@ge\endcsname{\let\PY@it=\textit}
\expandafter\def\csname PY@tok@gs\endcsname{\let\PY@bf=\textbf}
\expandafter\def\csname PY@tok@gp\endcsname{\let\PY@bf=\textbf\def\PY@tc##1{\textcolor[rgb]{0.00,0.00,0.50}{##1}}}
\expandafter\def\csname PY@tok@go\endcsname{\def\PY@tc##1{\textcolor[rgb]{0.53,0.53,0.53}{##1}}}
\expandafter\def\csname PY@tok@gt\endcsname{\def\PY@tc##1{\textcolor[rgb]{0.00,0.27,0.87}{##1}}}
\expandafter\def\csname PY@tok@err\endcsname{\def\PY@bc##1{\setlength{\fboxsep}{0pt}\fcolorbox[rgb]{1.00,0.00,0.00}{1,1,1}{\strut ##1}}}
\expandafter\def\csname PY@tok@kc\endcsname{\let\PY@bf=\textbf\def\PY@tc##1{\textcolor[rgb]{0.00,0.50,0.00}{##1}}}
\expandafter\def\csname PY@tok@kd\endcsname{\let\PY@bf=\textbf\def\PY@tc##1{\textcolor[rgb]{0.00,0.50,0.00}{##1}}}
\expandafter\def\csname PY@tok@kn\endcsname{\let\PY@bf=\textbf\def\PY@tc##1{\textcolor[rgb]{0.00,0.50,0.00}{##1}}}
\expandafter\def\csname PY@tok@kr\endcsname{\let\PY@bf=\textbf\def\PY@tc##1{\textcolor[rgb]{0.00,0.50,0.00}{##1}}}
\expandafter\def\csname PY@tok@bp\endcsname{\def\PY@tc##1{\textcolor[rgb]{0.00,0.50,0.00}{##1}}}
\expandafter\def\csname PY@tok@fm\endcsname{\def\PY@tc##1{\textcolor[rgb]{0.00,0.00,1.00}{##1}}}
\expandafter\def\csname PY@tok@vc\endcsname{\def\PY@tc##1{\textcolor[rgb]{0.10,0.09,0.49}{##1}}}
\expandafter\def\csname PY@tok@vg\endcsname{\def\PY@tc##1{\textcolor[rgb]{0.10,0.09,0.49}{##1}}}
\expandafter\def\csname PY@tok@vi\endcsname{\def\PY@tc##1{\textcolor[rgb]{0.10,0.09,0.49}{##1}}}
\expandafter\def\csname PY@tok@vm\endcsname{\def\PY@tc##1{\textcolor[rgb]{0.10,0.09,0.49}{##1}}}
\expandafter\def\csname PY@tok@sa\endcsname{\def\PY@tc##1{\textcolor[rgb]{0.73,0.13,0.13}{##1}}}
\expandafter\def\csname PY@tok@sb\endcsname{\def\PY@tc##1{\textcolor[rgb]{0.73,0.13,0.13}{##1}}}
\expandafter\def\csname PY@tok@sc\endcsname{\def\PY@tc##1{\textcolor[rgb]{0.73,0.13,0.13}{##1}}}
\expandafter\def\csname PY@tok@dl\endcsname{\def\PY@tc##1{\textcolor[rgb]{0.73,0.13,0.13}{##1}}}
\expandafter\def\csname PY@tok@s2\endcsname{\def\PY@tc##1{\textcolor[rgb]{0.73,0.13,0.13}{##1}}}
\expandafter\def\csname PY@tok@sh\endcsname{\def\PY@tc##1{\textcolor[rgb]{0.73,0.13,0.13}{##1}}}
\expandafter\def\csname PY@tok@s1\endcsname{\def\PY@tc##1{\textcolor[rgb]{0.73,0.13,0.13}{##1}}}
\expandafter\def\csname PY@tok@mb\endcsname{\def\PY@tc##1{\textcolor[rgb]{0.40,0.40,0.40}{##1}}}
\expandafter\def\csname PY@tok@mf\endcsname{\def\PY@tc##1{\textcolor[rgb]{0.40,0.40,0.40}{##1}}}
\expandafter\def\csname PY@tok@mh\endcsname{\def\PY@tc##1{\textcolor[rgb]{0.40,0.40,0.40}{##1}}}
\expandafter\def\csname PY@tok@mi\endcsname{\def\PY@tc##1{\textcolor[rgb]{0.40,0.40,0.40}{##1}}}
\expandafter\def\csname PY@tok@il\endcsname{\def\PY@tc##1{\textcolor[rgb]{0.40,0.40,0.40}{##1}}}
\expandafter\def\csname PY@tok@mo\endcsname{\def\PY@tc##1{\textcolor[rgb]{0.40,0.40,0.40}{##1}}}
\expandafter\def\csname PY@tok@ch\endcsname{\let\PY@it=\textit\def\PY@tc##1{\textcolor[rgb]{0.25,0.50,0.50}{##1}}}
\expandafter\def\csname PY@tok@cm\endcsname{\let\PY@it=\textit\def\PY@tc##1{\textcolor[rgb]{0.25,0.50,0.50}{##1}}}
\expandafter\def\csname PY@tok@cpf\endcsname{\let\PY@it=\textit\def\PY@tc##1{\textcolor[rgb]{0.25,0.50,0.50}{##1}}}
\expandafter\def\csname PY@tok@c1\endcsname{\let\PY@it=\textit\def\PY@tc##1{\textcolor[rgb]{0.25,0.50,0.50}{##1}}}
\expandafter\def\csname PY@tok@cs\endcsname{\let\PY@it=\textit\def\PY@tc##1{\textcolor[rgb]{0.25,0.50,0.50}{##1}}}

\def\PYZbs{\char`\\}
\def\PYZus{\char`\_}
\def\PYZob{\char`\{}
\def\PYZcb{\char`\}}
\def\PYZca{\char`\^}
\def\PYZam{\char`\&}
\def\PYZlt{\char`\<}
\def\PYZgt{\char`\>}
\def\PYZsh{\char`\#}
\def\PYZpc{\char`\%}
\def\PYZdl{\char`\$}
\def\PYZhy{\char`\-}
\def\PYZsq{\char`\'}
\def\PYZdq{\char`\"}
\def\PYZti{\char`\~}
% for compatibility with earlier versions
\def\PYZat{@}
\def\PYZlb{[}
\def\PYZrb{]}
\makeatother


    % Exact colors from NB
    \definecolor{incolor}{rgb}{0.0, 0.0, 0.5}
    \definecolor{outcolor}{rgb}{0.545, 0.0, 0.0}



    
    % Prevent overflowing lines due to hard-to-break entities
    \sloppy 
    % Setup hyperref package
    \hypersetup{
      breaklinks=true,  % so long urls are correctly broken across lines
      colorlinks=true,
      urlcolor=urlcolor,
      linkcolor=linkcolor,
      citecolor=citecolor,
      }
    % Slightly bigger margins than the latex defaults
    
    \geometry{verbose,tmargin=1in,bmargin=1in,lmargin=1in,rmargin=1in}
    
    

    \begin{document}
    
    
    \maketitle
    
    

    
    \subsection{1. Where are the old left-handed
people?}\label{where-are-the-old-left-handed-people}

Barack Obama is left-handed. So are Bill Gates and Oprah Winfrey; so
were Babe Ruth and Marie Curie. A 1991 study reported that left-handed
people die on average nine years earlier than right-handed people. Nine
years! Could this really be true?

In this notebook, we will explore this phenomenon using age distribution
data to see if we can reproduce a difference in average age at death
purely from the changing rates of left-handedness over time, refuting
the claim of early death for left-handers. This notebook uses pandas and
Bayesian statistics to analyze the probability of being a certain age at
death given that you are reported as left-handed or right-handed.

A National Geographic survey in 1986 resulted in over a million
responses that included age, sex, and hand preference for throwing and
writing. Researchers Avery Gilbert and Charles Wysocki analyzed this
data and noticed that rates of left-handedness were around 13\% for
people younger than 40 but decreased with age to about 5\% by the age of
80. They concluded based on analysis of a subgroup of people who throw
left-handed but write right-handed that this age-dependence was
primarily due to changing social acceptability of left-handedness. This
means that the rates aren't a factor of age specifically but rather of
the year you were born, and if the same study was done today, we should
expect a shifted version of the same distribution as a function of age.
Ultimately, we'll see what effect this changing rate has on the apparent
mean age of death of left-handed people, but let's start by plotting the
rates of left-handedness as a function of age.

This notebook uses two datasets: death distribution data for the United
States from the year 1999 (source website here) and rates of
left-handedness digitized from a figure in this 1992 paper by Gilbert
and Wysocki.

    \begin{Verbatim}[commandchars=\\\{\}]
{\color{incolor}In [{\color{incolor}19}]:} \PY{c+c1}{\PYZsh{} import libraries}
         \PY{k+kn}{import} \PY{n+nn}{pandas} \PY{k}{as} \PY{n+nn}{pd}
         \PY{k+kn}{import} \PY{n+nn}{matplotlib}\PY{n+nn}{.}\PY{n+nn}{pyplot} \PY{k}{as} \PY{n+nn}{plt}
         
         \PY{c+c1}{\PYZsh{} load the data}
         \PY{n}{data\PYZus{}url\PYZus{}1} \PY{o}{=} \PY{l+s+s2}{\PYZdq{}}\PY{l+s+s2}{https://gist.githubusercontent.com/mbonsma/8da0990b71ba9a09f7de395574e54df1/raw/aec88b30af87fad8d45da7e774223f91dad09e88/lh\PYZus{}data.csv}\PY{l+s+s2}{\PYZdq{}}
         \PY{n}{lefthanded\PYZus{}data} \PY{o}{=} \PY{n}{pd}\PY{o}{.}\PY{n}{read\PYZus{}csv}\PY{p}{(}\PY{n}{data\PYZus{}url\PYZus{}1}\PY{p}{)}
         
         \PY{c+c1}{\PYZsh{} plot male and female left\PYZhy{}handedness rates vs. age}
         \PY{o}{\PYZpc{}}\PY{k}{matplotlib} inline
         \PY{n}{fig}\PY{p}{,} \PY{n}{ax} \PY{o}{=} \PY{n}{plt}\PY{o}{.}\PY{n}{subplots}\PY{p}{(}\PY{p}{)} \PY{c+c1}{\PYZsh{} create figure and axis objects}
         \PY{n}{ax}\PY{o}{.}\PY{n}{plot}\PY{p}{(}\PY{l+s+s1}{\PYZsq{}}\PY{l+s+s1}{Age}\PY{l+s+s1}{\PYZsq{}}\PY{p}{,} \PY{l+s+s1}{\PYZsq{}}\PY{l+s+s1}{Female}\PY{l+s+s1}{\PYZsq{}}\PY{p}{,} \PY{n}{data}\PY{o}{=}\PY{n}{lefthanded\PYZus{}data}\PY{p}{,} \PY{n}{marker} \PY{o}{=} \PY{l+s+s1}{\PYZsq{}}\PY{l+s+s1}{o}\PY{l+s+s1}{\PYZsq{}}\PY{p}{)} \PY{c+c1}{\PYZsh{} plot \PYZdq{}Female\PYZdq{} vs. \PYZdq{}Age\PYZdq{}}
         \PY{n}{ax}\PY{o}{.}\PY{n}{plot}\PY{p}{(}\PY{l+s+s1}{\PYZsq{}}\PY{l+s+s1}{Age}\PY{l+s+s1}{\PYZsq{}}\PY{p}{,} \PY{l+s+s1}{\PYZsq{}}\PY{l+s+s1}{Male}\PY{l+s+s1}{\PYZsq{}}\PY{p}{,} \PY{n}{data}\PY{o}{=}\PY{n}{lefthanded\PYZus{}data}\PY{p}{,} \PY{n}{marker} \PY{o}{=} \PY{l+s+s1}{\PYZsq{}}\PY{l+s+s1}{x}\PY{l+s+s1}{\PYZsq{}}\PY{p}{)} \PY{c+c1}{\PYZsh{} plot \PYZdq{}Male\PYZdq{} vs. \PYZdq{}Age\PYZdq{}}
         \PY{n}{ax}\PY{o}{.}\PY{n}{legend}\PY{p}{(}\PY{p}{)} \PY{c+c1}{\PYZsh{} add a legend}
         \PY{n}{ax}\PY{o}{.}\PY{n}{set\PYZus{}xlabel}\PY{p}{(}\PY{l+s+s1}{\PYZsq{}}\PY{l+s+s1}{Age}\PY{l+s+s1}{\PYZsq{}}\PY{p}{)}
         \PY{n}{ax}\PY{o}{.}\PY{n}{set\PYZus{}ylabel}\PY{p}{(}\PY{l+s+s1}{\PYZsq{}}\PY{l+s+s1}{Left\PYZhy{}handedness Peference Rite}\PY{l+s+s1}{\PYZsq{}}\PY{p}{)}
         \PY{n}{plt}\PY{o}{.}\PY{n}{show}\PY{p}{(}\PY{p}{)}
\end{Verbatim}


    \begin{center}
    \adjustimage{max size={0.9\linewidth}{0.9\paperheight}}{output_1_0.png}
    \end{center}
    { \hspace*{\fill} \\}
    
    \subsection{2. Rates of left-handedness over
time}\label{rates-of-left-handedness-over-time}

Let's convert this data into a plot of the rates of left-handedness as a
function of the year of birth, and average over male and female to get a
single rate for both sexes.

Since the study was done in 1986, the data after this conversion will be
the percentage of people alive in 1986 who are left-handed as a function
of the year they were born.

    \begin{Verbatim}[commandchars=\\\{\}]
{\color{incolor}In [{\color{incolor}20}]:} \PY{c+c1}{\PYZsh{} create a new column for birth year of each age}
         \PY{n}{lefthanded\PYZus{}data}\PY{p}{[}\PY{l+s+s1}{\PYZsq{}}\PY{l+s+s1}{Birth\PYZus{}year}\PY{l+s+s1}{\PYZsq{}}\PY{p}{]} \PY{o}{=} \PY{l+m+mi}{1986} \PY{o}{\PYZhy{}} \PY{n}{lefthanded\PYZus{}data}\PY{p}{[}\PY{l+s+s1}{\PYZsq{}}\PY{l+s+s1}{Age}\PY{l+s+s1}{\PYZsq{}}\PY{p}{]}
         
         \PY{c+c1}{\PYZsh{} create a new column for the average of male and female}
         \PY{n}{lefthanded\PYZus{}data}\PY{p}{[}\PY{l+s+s1}{\PYZsq{}}\PY{l+s+s1}{Mean\PYZus{}lh}\PY{l+s+s1}{\PYZsq{}}\PY{p}{]} \PY{o}{=} \PY{n}{lefthanded\PYZus{}data}\PY{p}{[}\PY{p}{[}\PY{l+s+s1}{\PYZsq{}}\PY{l+s+s1}{Female}\PY{l+s+s1}{\PYZsq{}}\PY{p}{,} \PY{l+s+s1}{\PYZsq{}}\PY{l+s+s1}{Male}\PY{l+s+s1}{\PYZsq{}}\PY{p}{]}\PY{p}{]}\PY{o}{.}\PY{n}{mean}\PY{p}{(}\PY{n}{axis}\PY{o}{=}\PY{l+m+mi}{1}\PY{p}{)}
         
         \PY{c+c1}{\PYZsh{} create a plot of the \PYZsq{}Mean\PYZus{}lh\PYZsq{} column vs. \PYZsq{}Birth\PYZus{}year\PYZsq{}}
         \PY{n}{fig}\PY{p}{,} \PY{n}{ax} \PY{o}{=} \PY{n}{plt}\PY{o}{.}\PY{n}{subplots}\PY{p}{(}\PY{p}{)}
         \PY{n}{ax}\PY{o}{.}\PY{n}{plot}\PY{p}{(}\PY{l+s+s1}{\PYZsq{}}\PY{l+s+s1}{Birth\PYZus{}year}\PY{l+s+s1}{\PYZsq{}}\PY{p}{,} \PY{l+s+s1}{\PYZsq{}}\PY{l+s+s1}{Mean\PYZus{}lh}\PY{l+s+s1}{\PYZsq{}}\PY{p}{,} \PY{n}{data}\PY{o}{=}\PY{n}{lefthanded\PYZus{}data}\PY{p}{)} \PY{c+c1}{\PYZsh{} plot \PYZsq{}Mean\PYZus{}lh\PYZsq{} vs. \PYZsq{}Birth\PYZus{}year\PYZsq{}}
         \PY{n}{ax}\PY{o}{.}\PY{n}{set\PYZus{}xlabel}\PY{p}{(}\PY{l+s+s1}{\PYZsq{}}\PY{l+s+s1}{Year of Birth}\PY{l+s+s1}{\PYZsq{}}\PY{p}{)} \PY{c+c1}{\PYZsh{} set the x label for the plot}
         \PY{n}{ax}\PY{o}{.}\PY{n}{set\PYZus{}ylabel}\PY{p}{(}\PY{l+s+s1}{\PYZsq{}}\PY{l+s+s1}{Mean rate of left\PYZhy{}handedness}\PY{l+s+s1}{\PYZsq{}}\PY{p}{)} \PY{c+c1}{\PYZsh{} set the y label for the plot}
         \PY{n}{plt}\PY{o}{.}\PY{n}{show}\PY{p}{(}\PY{p}{)}
\end{Verbatim}


    \begin{center}
    \adjustimage{max size={0.9\linewidth}{0.9\paperheight}}{output_3_0.png}
    \end{center}
    { \hspace*{\fill} \\}
    
    \subsection{3. Applying Bayes' rule}\label{applying-bayes-rule}

The probability of dying at a certain age given that you're left-handed
is not equal to the probability of being left-handed given that you died
at a certain age. This inequality is why we need Bayes' theorem, a
statement about conditional probability which allows us to update our
beliefs after seeing evidence.

We want to calculate the probability of dying at age A given that you're
left-handed. Let's write this in shorthand as P(A \textbar{} LH). We
also want the same quantity for right-handers: P(A \textbar{} RH).

Here's Bayes' theorem for the two events we care about: left-handedness
(LH) and dying at age A.

\[P(A | LH) = \frac{P(LH|A) P(A)}{P(LH)}\]

where

\[P(LH | A) \text{ is the probability that you are left-handed given that you died at age A}\]

\[P(A) \text{ is the overall probability of dying at age A}\]

and

\[P(LH) \text{ is the overall probability of being left-handed}\]

We will now calculate each of these three quantities, beginning with
P(LH \textbar{} A).

To calculate P(LH \textbar{} A) for ages that might fall outside the
original data, we will need to extrapolate the data to earlier and later
years. Since the rates flatten out in the early 1900s and late 1900s,
we'll use a few points at each end and take the mean to extrapolate the
rates on each end. The number of points used for this is arbitrary, but
we'll pick 10 since the data looks flat-ish until about 1910.

    \begin{Verbatim}[commandchars=\\\{\}]
{\color{incolor}In [{\color{incolor}21}]:} \PY{c+c1}{\PYZsh{} import library}
         \PY{k+kn}{import} \PY{n+nn}{numpy} \PY{k}{as} \PY{n+nn}{np}
         
         \PY{c+c1}{\PYZsh{} create a function for P(LH | A)}
         \PY{k}{def} \PY{n+nf}{P\PYZus{}lh\PYZus{}given\PYZus{}A}\PY{p}{(}\PY{n}{ages\PYZus{}of\PYZus{}death}\PY{p}{,} \PY{n}{study\PYZus{}year} \PY{o}{=} \PY{l+m+mi}{1990}\PY{p}{)}\PY{p}{:}
             \PY{l+s+sd}{\PYZdq{}\PYZdq{}\PYZdq{} P(Left\PYZhy{}handed | ages of death), calculated based on the reported rates of left\PYZhy{}handedness.}
         \PY{l+s+sd}{    Inputs: numpy array of ages of death, study\PYZus{}year}
         \PY{l+s+sd}{    Returns: probability of left\PYZhy{}handedness given that subjects died in `study\PYZus{}year` at ages `ages\PYZus{}of\PYZus{}death` \PYZdq{}\PYZdq{}\PYZdq{}}
             
             \PY{c+c1}{\PYZsh{} Use the mean of the 10 last and 10 first points for left\PYZhy{}handedness rates before and after the start }
             \PY{n}{early\PYZus{}1900s\PYZus{}rate} \PY{o}{=} \PY{n}{lefthanded\PYZus{}data}\PY{p}{[}\PY{l+s+s1}{\PYZsq{}}\PY{l+s+s1}{Mean\PYZus{}lh}\PY{l+s+s1}{\PYZsq{}}\PY{p}{]}\PY{o}{.}\PY{n}{iloc}\PY{p}{[}\PY{o}{\PYZhy{}}\PY{l+m+mi}{10}\PY{p}{:}\PY{p}{]}\PY{o}{.}\PY{n}{mean}\PY{p}{(}\PY{p}{)}
             \PY{n}{late\PYZus{}1900s\PYZus{}rate} \PY{o}{=} \PY{n}{lefthanded\PYZus{}data}\PY{p}{[}\PY{l+s+s1}{\PYZsq{}}\PY{l+s+s1}{Mean\PYZus{}lh}\PY{l+s+s1}{\PYZsq{}}\PY{p}{]}\PY{o}{.}\PY{n}{iloc}\PY{p}{[}\PY{p}{:}\PY{l+m+mi}{10}\PY{p}{]}\PY{o}{.}\PY{n}{mean}\PY{p}{(}\PY{p}{)}
             \PY{n}{middle\PYZus{}rates} \PY{o}{=} \PY{n}{lefthanded\PYZus{}data}\PY{o}{.}\PY{n}{loc}\PY{p}{[}\PY{n}{lefthanded\PYZus{}data}\PY{p}{[}\PY{l+s+s1}{\PYZsq{}}\PY{l+s+s1}{Birth\PYZus{}year}\PY{l+s+s1}{\PYZsq{}}\PY{p}{]}\PY{o}{.}\PY{n}{isin}\PY{p}{(}\PY{n}{study\PYZus{}year} \PY{o}{\PYZhy{}} \PY{n}{ages\PYZus{}of\PYZus{}death}\PY{p}{)}\PY{p}{]}\PY{p}{[}\PY{l+s+s1}{\PYZsq{}}\PY{l+s+s1}{Mean\PYZus{}lh}\PY{l+s+s1}{\PYZsq{}}\PY{p}{]}
             \PY{n}{youngest\PYZus{}age} \PY{o}{=} \PY{n}{study\PYZus{}year} \PY{o}{\PYZhy{}} \PY{l+m+mi}{1986} \PY{o}{+} \PY{l+m+mi}{10} \PY{c+c1}{\PYZsh{} the youngest age is 10}
             \PY{n}{oldest\PYZus{}age} \PY{o}{=} \PY{n}{study\PYZus{}year} \PY{o}{\PYZhy{}} \PY{l+m+mi}{1986} \PY{o}{+} \PY{l+m+mi}{86} \PY{c+c1}{\PYZsh{} the oldest age is 86}
             
             \PY{n}{P\PYZus{}return} \PY{o}{=} \PY{n}{np}\PY{o}{.}\PY{n}{zeros}\PY{p}{(}\PY{n}{ages\PYZus{}of\PYZus{}death}\PY{o}{.}\PY{n}{shape}\PY{p}{)} \PY{c+c1}{\PYZsh{} create an empty array to store the results}
             \PY{c+c1}{\PYZsh{} extract rate of left\PYZhy{}handedness for people of ages \PYZsq{}ages\PYZus{}of\PYZus{}death\PYZsq{}}
             \PY{n}{P\PYZus{}return}\PY{p}{[}\PY{n}{ages\PYZus{}of\PYZus{}death} \PY{o}{\PYZgt{}} \PY{n}{oldest\PYZus{}age}\PY{p}{]} \PY{o}{=} \PY{n}{early\PYZus{}1900s\PYZus{}rate} \PY{o}{/} \PY{l+m+mi}{100}
             \PY{n}{P\PYZus{}return}\PY{p}{[}\PY{n}{ages\PYZus{}of\PYZus{}death} \PY{o}{\PYZlt{}} \PY{n}{youngest\PYZus{}age}\PY{p}{]} \PY{o}{=} \PY{n}{late\PYZus{}1900s\PYZus{}rate} \PY{o}{/} \PY{l+m+mi}{100}
             \PY{n}{P\PYZus{}return}\PY{p}{[}\PY{n}{np}\PY{o}{.}\PY{n}{logical\PYZus{}and}\PY{p}{(}\PY{p}{(}\PY{n}{ages\PYZus{}of\PYZus{}death} \PY{o}{\PYZlt{}}\PY{o}{=} \PY{n}{oldest\PYZus{}age}\PY{p}{)}\PY{p}{,} \PY{p}{(}\PY{n}{ages\PYZus{}of\PYZus{}death} \PY{o}{\PYZgt{}}\PY{o}{=} \PY{n}{youngest\PYZus{}age}\PY{p}{)}\PY{p}{)}\PY{p}{]} \PY{o}{=} \PY{n}{middle\PYZus{}rates} \PY{o}{/} \PY{l+m+mi}{100}
             
             \PY{k}{return} \PY{n}{P\PYZus{}return}
\end{Verbatim}


    \subsection{4. When do people normally
die?}\label{when-do-people-normally-die}

To estimate the probability of living to an age A, we can use data that
gives the number of people who died in a given year and how old they
were to create a distribution of ages of death. If we normalize the
numbers to the total number of people who died, we can think of this
data as a probability distribution that gives the probability of dying
at age A. The data we'll use for this is from the entire US for the year
1999 - the closest I could find for the time range we're interested in.

In this block, we'll load in the death distribution data and plot it.
The first column is the age, and the other columns are the number of
people who died at that age.

    \begin{Verbatim}[commandchars=\\\{\}]
{\color{incolor}In [{\color{incolor}22}]:} \PY{c+c1}{\PYZsh{} Death distribution data for the United States in 1999}
         \PY{n}{data\PYZus{}url\PYZus{}2} \PY{o}{=} \PY{l+s+s2}{\PYZdq{}}\PY{l+s+s2}{https://gist.githubusercontent.com/mbonsma/2f4076aab6820ca1807f4e29f75f18ec/raw/62f3ec07514c7e31f5979beeca86f19991540796/cdc\PYZus{}vs00199\PYZus{}table310.tsv}\PY{l+s+s2}{\PYZdq{}}
         
         \PY{c+c1}{\PYZsh{} load death distribution data}
         \PY{n}{death\PYZus{}distribution\PYZus{}data} \PY{o}{=} \PY{n}{pd}\PY{o}{.}\PY{n}{read\PYZus{}csv}\PY{p}{(}\PY{n}{data\PYZus{}url\PYZus{}2}\PY{p}{,} \PY{n}{sep}\PY{o}{=}\PY{l+s+s1}{\PYZsq{}}\PY{l+s+se}{\PYZbs{}t}\PY{l+s+s1}{\PYZsq{}}\PY{p}{,} \PY{n}{skiprows}\PY{o}{=}\PY{p}{[}\PY{l+m+mi}{1}\PY{p}{]}\PY{p}{)}
         
         \PY{c+c1}{\PYZsh{} drop NaN values from the `Both Sexes` column}
         \PY{n}{death\PYZus{}distribution\PYZus{}data} \PY{o}{=} \PY{n}{death\PYZus{}distribution\PYZus{}data}\PY{o}{.}\PY{n}{dropna}\PY{p}{(}\PY{n}{subset}\PY{o}{=}\PY{p}{[}\PY{l+s+s1}{\PYZsq{}}\PY{l+s+s1}{Both Sexes}\PY{l+s+s1}{\PYZsq{}}\PY{p}{]}\PY{p}{)}
         
         \PY{c+c1}{\PYZsh{} plot number of people who died as a function of age}
         \PY{n}{fig}\PY{p}{,} \PY{n}{ax} \PY{o}{=} \PY{n}{plt}\PY{o}{.}\PY{n}{subplots}\PY{p}{(}\PY{p}{)}
         \PY{n}{ax}\PY{o}{.}\PY{n}{plot}\PY{p}{(}\PY{l+s+s1}{\PYZsq{}}\PY{l+s+s1}{Age}\PY{l+s+s1}{\PYZsq{}}\PY{p}{,} \PY{l+s+s1}{\PYZsq{}}\PY{l+s+s1}{Both Sexes}\PY{l+s+s1}{\PYZsq{}}\PY{p}{,} \PY{n}{data} \PY{o}{=} \PY{n}{death\PYZus{}distribution\PYZus{}data}\PY{p}{,} \PY{n}{marker}\PY{o}{=}\PY{l+s+s1}{\PYZsq{}}\PY{l+s+s1}{o}\PY{l+s+s1}{\PYZsq{}}\PY{p}{)} \PY{c+c1}{\PYZsh{} plot \PYZsq{}Both Sexes\PYZsq{} vs. \PYZsq{}Age\PYZsq{}}
         \PY{n}{ax}\PY{o}{.}\PY{n}{set\PYZus{}xlabel}\PY{p}{(}\PY{l+s+s1}{\PYZsq{}}\PY{l+s+s1}{Age}\PY{l+s+s1}{\PYZsq{}}\PY{p}{)} 
         \PY{n}{ax}\PY{o}{.}\PY{n}{set\PYZus{}ylabel}\PY{p}{(}\PY{l+s+s1}{\PYZsq{}}\PY{l+s+s1}{Number of people dead in both sexes}\PY{l+s+s1}{\PYZsq{}}\PY{p}{)}
         \PY{n}{plt}\PY{o}{.}\PY{n}{show}\PY{p}{(}\PY{p}{)}
\end{Verbatim}


    \begin{center}
    \adjustimage{max size={0.9\linewidth}{0.9\paperheight}}{output_7_0.png}
    \end{center}
    { \hspace*{\fill} \\}
    
    \subsection{5. The overall probability of
left-handedness}\label{the-overall-probability-of-left-handedness}

In the previous code block we loaded data to give us P(A), and now we
need P(LH). P(LH) is the probability that a person who died in our
particular study year is left-handed, assuming we know nothing else
about them. This is the average left-handedness in the population of
deceased people, and we can calculate it by summing up all of the
left-handedness probabilities for each age, weighted with the number of
deceased people at each age, then divided by the total number of
deceased people to get a probability. In equation form, this is what
we're calculating, where N(A) is the number of people who died at age A
(given by the dataframe death\_distribution\_data):

    \begin{Verbatim}[commandchars=\\\{\}]
{\color{incolor}In [{\color{incolor}23}]:} \PY{k}{def} \PY{n+nf}{P\PYZus{}lh}\PY{p}{(}\PY{n}{death\PYZus{}distribution\PYZus{}data}\PY{p}{,} \PY{n}{study\PYZus{}year} \PY{o}{=} \PY{l+m+mi}{1990}\PY{p}{)}\PY{p}{:} \PY{c+c1}{\PYZsh{} sum over P\PYZus{}lh for each age group}
             \PY{l+s+sd}{\PYZdq{}\PYZdq{}\PYZdq{} Overall probability of being left\PYZhy{}handed if you died in the study year}
         \PY{l+s+sd}{    Input: dataframe of death distribution data, study year}
         \PY{l+s+sd}{    Output: P(LH), a single floating point number \PYZdq{}\PYZdq{}\PYZdq{}}
             \PY{n}{p\PYZus{}list} \PY{o}{=} \PY{n}{death\PYZus{}distribution\PYZus{}data}\PY{p}{[}\PY{l+s+s1}{\PYZsq{}}\PY{l+s+s1}{Both Sexes}\PY{l+s+s1}{\PYZsq{}}\PY{p}{]} \PY{o}{*} \PY{n}{P\PYZus{}lh\PYZus{}given\PYZus{}A}\PY{p}{(}\PY{n}{death\PYZus{}distribution\PYZus{}data}\PY{p}{[}\PY{l+s+s1}{\PYZsq{}}\PY{l+s+s1}{Age}\PY{l+s+s1}{\PYZsq{}}\PY{p}{]}\PY{p}{,} \PY{n}{study\PYZus{}year}\PY{p}{)} \PY{c+c1}{\PYZsh{} multiply number of dead people by P\PYZus{}lh\PYZus{}given\PYZus{}A}
             \PY{n}{p} \PY{o}{=}  \PY{n}{p\PYZus{}list}\PY{o}{.}\PY{n}{sum}\PY{p}{(}\PY{p}{)} \PY{c+c1}{\PYZsh{} calculate the sum of p\PYZus{}list}
             \PY{k}{return} \PY{n}{p}\PY{o}{/}\PY{n}{death\PYZus{}distribution\PYZus{}data}\PY{p}{[}\PY{l+s+s1}{\PYZsq{}}\PY{l+s+s1}{Both Sexes}\PY{l+s+s1}{\PYZsq{}}\PY{p}{]}\PY{o}{.}\PY{n}{sum}\PY{p}{(}\PY{p}{)} \PY{c+c1}{\PYZsh{} normalize to total number of people (sum of death\PYZus{}distribution\PYZus{}data[\PYZsq{}Both Sexes\PYZsq{}])}
         
         \PY{n+nb}{print}\PY{p}{(}\PY{n}{P\PYZus{}lh}\PY{p}{(}\PY{n}{death\PYZus{}distribution\PYZus{}data}\PY{p}{,} \PY{n}{study\PYZus{}year}\PY{o}{=}\PY{l+m+mi}{1990}\PY{p}{)}\PY{p}{)}
\end{Verbatim}


    \begin{Verbatim}[commandchars=\\\{\}]
0.07766387615350638

    \end{Verbatim}

    \subsection{6. Putting it all together: dying while left-handed
(i)}\label{putting-it-all-together-dying-while-left-handed-i}

Now we have the means of calculating all three quantities we need: P(A),
P(LH), and P(LH \textbar{} A). We can combine all three using Bayes'
rule to get P(A \textbar{} LH), the probability of being age A at death
(in the study year) given that you're left-handed. To make this answer
meaningful, though, we also want to compare it to P(A \textbar{} RH),
the probability of being age A at death given that you're right-handed.

We're calculating the following quantity twice, once for left-handers
and once for right-handers.

\[P(A | LH) = \frac{P(LH|A) P(A)}{P(LH)}\]

First, for left-handers.

    \begin{Verbatim}[commandchars=\\\{\}]
{\color{incolor}In [{\color{incolor}24}]:} \PY{k}{def} \PY{n+nf}{P\PYZus{}A\PYZus{}given\PYZus{}lh}\PY{p}{(}\PY{n}{ages\PYZus{}of\PYZus{}death}\PY{p}{,} \PY{n}{death\PYZus{}distribution\PYZus{}data}\PY{p}{,} \PY{n}{study\PYZus{}year} \PY{o}{=} \PY{l+m+mi}{1990}\PY{p}{)}\PY{p}{:}
             \PY{l+s+sd}{\PYZdq{}\PYZdq{}\PYZdq{} The overall probability of being a particular `age\PYZus{}of\PYZus{}death` given that you\PYZsq{}re left\PYZhy{}handed \PYZdq{}\PYZdq{}\PYZdq{}}
             \PY{n}{P\PYZus{}A} \PY{o}{=} \PY{n}{death\PYZus{}distribution\PYZus{}data}\PY{p}{[}\PY{l+s+s1}{\PYZsq{}}\PY{l+s+s1}{Both Sexes}\PY{l+s+s1}{\PYZsq{}}\PY{p}{]}\PY{p}{[}\PY{n}{ages\PYZus{}of\PYZus{}death}\PY{p}{]} \PY{o}{/} \PY{n}{death\PYZus{}distribution\PYZus{}data}\PY{p}{[}\PY{l+s+s1}{\PYZsq{}}\PY{l+s+s1}{Both Sexes}\PY{l+s+s1}{\PYZsq{}}\PY{p}{]}\PY{o}{.}\PY{n}{sum}\PY{p}{(}\PY{p}{)}
             \PY{n}{P\PYZus{}left} \PY{o}{=} \PY{n}{P\PYZus{}lh}\PY{p}{(}\PY{n}{death\PYZus{}distribution\PYZus{}data}\PY{p}{,} \PY{n}{study\PYZus{}year}\PY{p}{)} \PY{c+c1}{\PYZsh{} use P\PYZus{}lh function to get probability of left\PYZhy{}handedness overall}
             \PY{n}{P\PYZus{}lh\PYZus{}A} \PY{o}{=} \PY{n}{P\PYZus{}lh\PYZus{}given\PYZus{}A}\PY{p}{(}\PY{n}{ages\PYZus{}of\PYZus{}death}\PY{p}{,} \PY{n}{study\PYZus{}year}\PY{p}{)} \PY{c+c1}{\PYZsh{} use P\PYZus{}lh\PYZus{}given\PYZus{}A to get probability of left\PYZhy{}handedness for a certain age}
             \PY{k}{return} \PY{n}{P\PYZus{}lh\PYZus{}A}\PY{o}{*}\PY{n}{P\PYZus{}A}\PY{o}{/}\PY{n}{P\PYZus{}left}
\end{Verbatim}


    \subsection{7. Putting it all together: dying while left-handed
(ii)}\label{putting-it-all-together-dying-while-left-handed-ii}

And now for right-handers.

    \begin{Verbatim}[commandchars=\\\{\}]
{\color{incolor}In [{\color{incolor}25}]:} \PY{k}{def} \PY{n+nf}{P\PYZus{}A\PYZus{}given\PYZus{}rh}\PY{p}{(}\PY{n}{ages\PYZus{}of\PYZus{}death}\PY{p}{,} \PY{n}{death\PYZus{}distribution\PYZus{}data}\PY{p}{,} \PY{n}{study\PYZus{}year} \PY{o}{=} \PY{l+m+mi}{1990}\PY{p}{)}\PY{p}{:}
             \PY{l+s+sd}{\PYZdq{}\PYZdq{}\PYZdq{} The overall probability of being a particular }
         \PY{l+s+sd}{        `age\PYZus{}of\PYZus{}death` given that you\PYZsq{}re right\PYZhy{}handed \PYZdq{}\PYZdq{}\PYZdq{}}
             \PY{n}{P\PYZus{}A} \PY{o}{=} \PY{n}{death\PYZus{}distribution\PYZus{}data}\PY{p}{[}\PY{l+s+s1}{\PYZsq{}}\PY{l+s+s1}{Both Sexes}\PY{l+s+s1}{\PYZsq{}}\PY{p}{]}\PY{p}{[}\PY{n}{ages\PYZus{}of\PYZus{}death}\PY{p}{]} \PY{o}{/} \PY{n}{death\PYZus{}distribution\PYZus{}data}\PY{p}{[}\PY{l+s+s1}{\PYZsq{}}\PY{l+s+s1}{Both Sexes}\PY{l+s+s1}{\PYZsq{}}\PY{p}{]}\PY{o}{.}\PY{n}{sum}\PY{p}{(}\PY{p}{)}
             \PY{n}{P\PYZus{}right} \PY{o}{=} \PY{l+m+mi}{1} \PY{o}{\PYZhy{}} \PY{n}{P\PYZus{}lh}\PY{p}{(}\PY{n}{death\PYZus{}distribution\PYZus{}data}\PY{p}{,} \PY{n}{study\PYZus{}year}\PY{p}{)} \PY{c+c1}{\PYZsh{} either you\PYZsq{}re left\PYZhy{}handed or right\PYZhy{}handed, so P\PYZus{}right = 1 \PYZhy{} P\PYZus{}left}
             \PY{n}{P\PYZus{}rh\PYZus{}A} \PY{o}{=} \PY{l+m+mi}{1} \PY{o}{\PYZhy{}} \PY{n}{P\PYZus{}lh\PYZus{}given\PYZus{}A}\PY{p}{(}\PY{n}{ages\PYZus{}of\PYZus{}death}\PY{p}{,} \PY{n}{study\PYZus{}year}\PY{p}{)} \PY{c+c1}{\PYZsh{} P\PYZus{}rh\PYZus{}A = 1 \PYZhy{} P\PYZus{}lh\PYZus{}A }
             \PY{k}{return} \PY{n}{P\PYZus{}rh\PYZus{}A}\PY{o}{*}\PY{n}{P\PYZus{}A}\PY{o}{/}\PY{n}{P\PYZus{}right}
\end{Verbatim}


    \subsection{8. Plotting the distributions of conditional
probabilities}\label{plotting-the-distributions-of-conditional-probabilities}

Now that we have functions to calculate the probability of being age A
at death given that you're left-handed or right-handed, let's plot these
probabilities for a range of ages of death from 6 to 120.

Notice that the left-handed distribution has a bump below age 70: of the
pool of deceased people, left-handed people are more likely to be
younger.

    \begin{Verbatim}[commandchars=\\\{\}]
{\color{incolor}In [{\color{incolor}27}]:} \PY{n}{ages} \PY{o}{=} \PY{n}{np}\PY{o}{.}\PY{n}{arange}\PY{p}{(}\PY{l+m+mi}{6}\PY{p}{,} \PY{l+m+mi}{115}\PY{p}{,} \PY{l+m+mi}{1}\PY{p}{)} \PY{c+c1}{\PYZsh{} make a list of ages of death to plot}
         
         \PY{c+c1}{\PYZsh{} calculate the probability of being left\PYZhy{} or right\PYZhy{}handed for each }
         \PY{n}{left\PYZus{}handed\PYZus{}probability} \PY{o}{=} \PY{n}{P\PYZus{}A\PYZus{}given\PYZus{}lh}\PY{p}{(}\PY{n}{ages}\PY{p}{,} \PY{n}{death\PYZus{}distribution\PYZus{}data}\PY{p}{)}
         \PY{n}{right\PYZus{}handed\PYZus{}probability} \PY{o}{=} \PY{n}{P\PYZus{}A\PYZus{}given\PYZus{}rh}\PY{p}{(}\PY{n}{ages}\PY{p}{,} \PY{n}{death\PYZus{}distribution\PYZus{}data}\PY{p}{)}
         
         \PY{c+c1}{\PYZsh{} create a plot of the two probabilities vs. age}
         \PY{n}{fig}\PY{p}{,} \PY{n}{ax} \PY{o}{=} \PY{n}{plt}\PY{o}{.}\PY{n}{subplots}\PY{p}{(}\PY{p}{)} \PY{c+c1}{\PYZsh{} create figure and axis objects}
         \PY{n}{ax}\PY{o}{.}\PY{n}{plot}\PY{p}{(}\PY{n}{ages}\PY{p}{,} \PY{n}{left\PYZus{}handed\PYZus{}probability}\PY{p}{,} \PY{n}{label} \PY{o}{=} \PY{l+s+s2}{\PYZdq{}}\PY{l+s+s2}{Left\PYZhy{}handed}\PY{l+s+s2}{\PYZdq{}}\PY{p}{)}
         \PY{n}{ax}\PY{o}{.}\PY{n}{plot}\PY{p}{(}\PY{n}{ages}\PY{p}{,} \PY{n}{right\PYZus{}handed\PYZus{}probability}\PY{p}{,} \PY{n}{label} \PY{o}{=} \PY{l+s+s1}{\PYZsq{}}\PY{l+s+s1}{Right\PYZhy{}handed}\PY{l+s+s1}{\PYZsq{}}\PY{p}{)}
         \PY{n}{ax}\PY{o}{.}\PY{n}{legend}\PY{p}{(}\PY{p}{)} \PY{c+c1}{\PYZsh{} add a legend}
         \PY{n}{ax}\PY{o}{.}\PY{n}{set\PYZus{}xlabel}\PY{p}{(}\PY{l+s+s2}{\PYZdq{}}\PY{l+s+s2}{Age at death}\PY{l+s+s2}{\PYZdq{}}\PY{p}{)}
         \PY{n}{ax}\PY{o}{.}\PY{n}{set\PYZus{}ylabel}\PY{p}{(}\PY{l+s+sa}{r}\PY{l+s+s2}{\PYZdq{}}\PY{l+s+s2}{Probability of being age A at death}\PY{l+s+s2}{\PYZdq{}}\PY{p}{)}
         \PY{n}{plt}\PY{o}{.}\PY{n}{show}\PY{p}{(}\PY{p}{)}
\end{Verbatim}


    \begin{center}
    \adjustimage{max size={0.9\linewidth}{0.9\paperheight}}{output_15_0.png}
    \end{center}
    { \hspace*{\fill} \\}
    
    \subsection{9. Moment of truth: age of left and right-handers at
death}\label{moment-of-truth-age-of-left-and-right-handers-at-death}

Finally, let's compare our results with the original study that found
that left-handed people were nine years younger at death on average. We
can do this by calculating the mean of these probability distributions
in the same way we calculated P(LH) earlier, weighting the probability
distribution by age and summing over the result.

\[\text{Average age of left-handed people at death} = \sum_A A P(A | LH)\]

\[\text{Average age of right-handed people at death} = \sum_A A P(A | RH)\]

    \begin{Verbatim}[commandchars=\\\{\}]
{\color{incolor}In [{\color{incolor}11}]:} \PY{c+c1}{\PYZsh{} calculate average ages for left\PYZhy{}handed and right\PYZhy{}handed groups}
         \PY{c+c1}{\PYZsh{} use np.array so that two arrays can be multiplied}
         \PY{n}{average\PYZus{}lh\PYZus{}age} \PY{o}{=}  \PY{n}{np}\PY{o}{.}\PY{n}{nansum}\PY{p}{(}\PY{n}{ages}\PY{o}{*}\PY{n}{np}\PY{o}{.}\PY{n}{array}\PY{p}{(}\PY{n}{left\PYZus{}handed\PYZus{}probability}\PY{p}{)}\PY{p}{)}
         \PY{n}{average\PYZus{}rh\PYZus{}age} \PY{o}{=}  \PY{n}{np}\PY{o}{.}\PY{n}{nansum}\PY{p}{(}\PY{n}{ages}\PY{o}{*}\PY{n}{np}\PY{o}{.}\PY{n}{array}\PY{p}{(}\PY{n}{right\PYZus{}handed\PYZus{}probability}\PY{p}{)}\PY{p}{)}
         
         \PY{c+c1}{\PYZsh{} print the average ages for each group}
         \PY{n+nb}{print}\PY{p}{(}\PY{l+s+s1}{\PYZsq{}}\PY{l+s+s1}{Average left\PYZhy{}handed death age is }\PY{l+s+s1}{\PYZsq{}}\PY{o}{+} \PY{n+nb}{str}\PY{p}{(}\PY{n}{average\PYZus{}lh\PYZus{}age}\PY{p}{)}\PY{p}{)}
         \PY{n+nb}{print}\PY{p}{(}\PY{l+s+s1}{\PYZsq{}}\PY{l+s+s1}{Average right\PYZhy{}handed death age is }\PY{l+s+s1}{\PYZsq{}}\PY{o}{+} \PY{n+nb}{str}\PY{p}{(}\PY{n}{average\PYZus{}rh\PYZus{}age}\PY{p}{)}\PY{p}{)}
         
         \PY{c+c1}{\PYZsh{} print the difference between the average ages}
         \PY{n+nb}{print}\PY{p}{(}\PY{l+s+s2}{\PYZdq{}}\PY{l+s+s2}{The difference in average ages is }\PY{l+s+s2}{\PYZdq{}} \PY{o}{+} 
               \PY{n+nb}{str}\PY{p}{(}\PY{n+nb}{round}\PY{p}{(}\PY{n}{average\PYZus{}lh\PYZus{}age} \PY{o}{\PYZhy{}} \PY{n}{average\PYZus{}rh\PYZus{}age}\PY{p}{,} \PY{l+m+mi}{1}\PY{p}{)}\PY{p}{)} \PY{o}{+} \PY{l+s+s2}{\PYZdq{}}\PY{l+s+s2}{ years.}\PY{l+s+s2}{\PYZdq{}}\PY{p}{)}
\end{Verbatim}


    \begin{Verbatim}[commandchars=\\\{\}]
Average left-handed death age is 67.24503662801027
Average right-handed death age is 72.79171936526477
The difference in average ages is -5.5 years.

    \end{Verbatim}

    \subsection{10. Final comments}\label{final-comments}

We got a pretty big age gap between left-handed and right-handed people
purely as a result of the changing rates of left-handedness in the
population, which is good news for left-handers: you probably won't die
young because of your sinisterness. The reported rates of
left-handedness have increased from just 3\% in the early 1900s to about
11\% today, which means that older people are much more likely to be
reported as right-handed than left-handed, and so looking at a sample of
recently deceased people will have more old right-handers.

Our number is still less than the 9-year gap measured in the study. It's
possible that some of the approximations we made are the cause:

We used death distribution data from almost ten years after the study
(1999 instead of 1991), and we used death data from the entire United
States instead of California alone (which was the original study).

We extrapolated the left-handedness survey results to older and younger
age groups, but it's possible our extrapolation wasn't close enough to
the true rates for those ages.

One thing we could do next is figure out how much variability we would
expect to encounter in the age difference purely because of random
sampling: if you take a smaller sample of recently deceased people and
assign handedness with the probabilities of the survey, what does that
distribution look like? How often would we encounter an age gap of nine
years using the same data and assumptions? We won't do that here, but
it's possible with this data and the tools of random sampling.

To finish off, let's calculate the age gap we'd expect if we did the
study in 2018 instead of in 1990. The gap turns out to be much smaller
since rates of left-handedness haven't increased for people born after
about 1960. Both the National Geographic study and the 1990 study
happened at a unique time - the rates of left-handedness had been
changing across the lifetimes of most people alive, and the difference
in handedness between old and young was at its most striking.

    \begin{Verbatim}[commandchars=\\\{\}]
{\color{incolor}In [{\color{incolor}12}]:} \PY{c+c1}{\PYZsh{} Calculate the probability of being left\PYZhy{} or right\PYZhy{}handed for all ages}
         \PY{n}{left\PYZus{}handed\PYZus{}probability\PYZus{}2018} \PY{o}{=}  \PY{n}{P\PYZus{}A\PYZus{}given\PYZus{}lh}\PY{p}{(}\PY{n}{ages}\PY{p}{,} \PY{n}{death\PYZus{}distribution\PYZus{}data}\PY{p}{,} \PY{n}{study\PYZus{}year}\PY{o}{=}\PY{l+m+mi}{2018}\PY{p}{)}
         \PY{n}{right\PYZus{}handed\PYZus{}probability\PYZus{}2018} \PY{o}{=} \PY{n}{P\PYZus{}A\PYZus{}given\PYZus{}rh}\PY{p}{(}\PY{n}{ages}\PY{p}{,} \PY{n}{death\PYZus{}distribution\PYZus{}data}\PY{p}{,} \PY{n}{study\PYZus{}year}\PY{o}{=}\PY{l+m+mi}{2018}\PY{p}{)}
         
         \PY{c+c1}{\PYZsh{} calculate average ages for left\PYZhy{}handed and right\PYZhy{}handed groups}
         \PY{n}{average\PYZus{}lh\PYZus{}age\PYZus{}2018} \PY{o}{=} \PY{n}{np}\PY{o}{.}\PY{n}{nansum}\PY{p}{(}\PY{n}{ages}\PY{o}{*}\PY{n}{np}\PY{o}{.}\PY{n}{array}\PY{p}{(}\PY{n}{left\PYZus{}handed\PYZus{}probability\PYZus{}2018}\PY{p}{)}\PY{p}{)}
         \PY{n}{average\PYZus{}rh\PYZus{}age\PYZus{}2018} \PY{o}{=} \PY{n}{np}\PY{o}{.}\PY{n}{nansum}\PY{p}{(}\PY{n}{ages}\PY{o}{*}\PY{n}{np}\PY{o}{.}\PY{n}{array}\PY{p}{(}\PY{n}{right\PYZus{}handed\PYZus{}probability\PYZus{}2018}\PY{p}{)}\PY{p}{)}
         
         \PY{n+nb}{print}\PY{p}{(}\PY{l+s+s2}{\PYZdq{}}\PY{l+s+s2}{The difference in average ages between left\PYZhy{}handed and right\PYZhy{}handed is }\PY{l+s+s2}{\PYZdq{}} \PY{o}{+} 
               \PY{n+nb}{str}\PY{p}{(}\PY{n+nb}{round}\PY{p}{(}\PY{n}{average\PYZus{}rh\PYZus{}age\PYZus{}2018} \PY{o}{\PYZhy{}} \PY{n}{average\PYZus{}lh\PYZus{}age\PYZus{}2018}\PY{p}{,} \PY{l+m+mi}{1}\PY{p}{)}\PY{p}{)} \PY{o}{+} \PY{l+s+s2}{\PYZdq{}}\PY{l+s+s2}{ years.}\PY{l+s+s2}{\PYZdq{}}\PY{p}{)}
\end{Verbatim}


    \begin{Verbatim}[commandchars=\\\{\}]
The difference in average ages between left-handed and right-handed is 2.3 years.

    \end{Verbatim}

    \begin{Verbatim}[commandchars=\\\{\}]
{\color{incolor}In [{\color{incolor}6}]:} 
\end{Verbatim}


    \begin{Verbatim}[commandchars=\\\{\}]

          File "<ipython-input-6-55abf84c2bf8>", line 1
        ((*- extends 'article.tplx' -*))
                                  \^{}
    SyntaxError: invalid syntax


    \end{Verbatim}


    % Add a bibliography block to the postdoc
    
    
    
    \end{document}
