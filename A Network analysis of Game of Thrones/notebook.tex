
% Default to the notebook output style

    


% Inherit from the specified cell style.




    
\documentclass[11pt]{article}

    
    
    \usepackage[T1]{fontenc}
    % Nicer default font (+ math font) than Computer Modern for most use cases
    \usepackage{mathpazo}

    % Basic figure setup, for now with no caption control since it's done
    % automatically by Pandoc (which extracts ![](path) syntax from Markdown).
    \usepackage{graphicx}
    % We will generate all images so they have a width \maxwidth. This means
    % that they will get their normal width if they fit onto the page, but
    % are scaled down if they would overflow the margins.
    \makeatletter
    \def\maxwidth{\ifdim\Gin@nat@width>\linewidth\linewidth
    \else\Gin@nat@width\fi}
    \makeatother
    \let\Oldincludegraphics\includegraphics
    % Set max figure width to be 80% of text width, for now hardcoded.
    \renewcommand{\includegraphics}[1]{\Oldincludegraphics[width=.8\maxwidth]{#1}}
    % Ensure that by default, figures have no caption (until we provide a
    % proper Figure object with a Caption API and a way to capture that
    % in the conversion process - todo).
    \usepackage{caption}
    \DeclareCaptionLabelFormat{nolabel}{}
    \captionsetup{labelformat=nolabel}

    \usepackage{adjustbox} % Used to constrain images to a maximum size 
    \usepackage{xcolor} % Allow colors to be defined
    \usepackage{enumerate} % Needed for markdown enumerations to work
    \usepackage{geometry} % Used to adjust the document margins
    \usepackage{amsmath} % Equations
    \usepackage{amssymb} % Equations
    \usepackage{textcomp} % defines textquotesingle
    % Hack from http://tex.stackexchange.com/a/47451/13684:
    \AtBeginDocument{%
        \def\PYZsq{\textquotesingle}% Upright quotes in Pygmentized code
    }
    \usepackage{upquote} % Upright quotes for verbatim code
    \usepackage{eurosym} % defines \euro
    \usepackage[mathletters]{ucs} % Extended unicode (utf-8) support
    \usepackage[utf8x]{inputenc} % Allow utf-8 characters in the tex document
    \usepackage{fancyvrb} % verbatim replacement that allows latex
    \usepackage{grffile} % extends the file name processing of package graphics 
                         % to support a larger range 
    % The hyperref package gives us a pdf with properly built
    % internal navigation ('pdf bookmarks' for the table of contents,
    % internal cross-reference links, web links for URLs, etc.)
    \usepackage{hyperref}
    \usepackage{longtable} % longtable support required by pandoc >1.10
    \usepackage{booktabs}  % table support for pandoc > 1.12.2
    \usepackage[inline]{enumitem} % IRkernel/repr support (it uses the enumerate* environment)
    \usepackage[normalem]{ulem} % ulem is needed to support strikethroughs (\sout)
                                % normalem makes italics be italics, not underlines
    

    
    
    % Colors for the hyperref package
    \definecolor{urlcolor}{rgb}{0,.145,.698}
    \definecolor{linkcolor}{rgb}{.71,0.21,0.01}
    \definecolor{citecolor}{rgb}{.12,.54,.11}

    % ANSI colors
    \definecolor{ansi-black}{HTML}{3E424D}
    \definecolor{ansi-black-intense}{HTML}{282C36}
    \definecolor{ansi-red}{HTML}{E75C58}
    \definecolor{ansi-red-intense}{HTML}{B22B31}
    \definecolor{ansi-green}{HTML}{00A250}
    \definecolor{ansi-green-intense}{HTML}{007427}
    \definecolor{ansi-yellow}{HTML}{DDB62B}
    \definecolor{ansi-yellow-intense}{HTML}{B27D12}
    \definecolor{ansi-blue}{HTML}{208FFB}
    \definecolor{ansi-blue-intense}{HTML}{0065CA}
    \definecolor{ansi-magenta}{HTML}{D160C4}
    \definecolor{ansi-magenta-intense}{HTML}{A03196}
    \definecolor{ansi-cyan}{HTML}{60C6C8}
    \definecolor{ansi-cyan-intense}{HTML}{258F8F}
    \definecolor{ansi-white}{HTML}{C5C1B4}
    \definecolor{ansi-white-intense}{HTML}{A1A6B2}

    % commands and environments needed by pandoc snippets
    % extracted from the output of `pandoc -s`
    \providecommand{\tightlist}{%
      \setlength{\itemsep}{0pt}\setlength{\parskip}{0pt}}
    \DefineVerbatimEnvironment{Highlighting}{Verbatim}{commandchars=\\\{\}}
    % Add ',fontsize=\small' for more characters per line
    \newenvironment{Shaded}{}{}
    \newcommand{\KeywordTok}[1]{\textcolor[rgb]{0.00,0.44,0.13}{\textbf{{#1}}}}
    \newcommand{\DataTypeTok}[1]{\textcolor[rgb]{0.56,0.13,0.00}{{#1}}}
    \newcommand{\DecValTok}[1]{\textcolor[rgb]{0.25,0.63,0.44}{{#1}}}
    \newcommand{\BaseNTok}[1]{\textcolor[rgb]{0.25,0.63,0.44}{{#1}}}
    \newcommand{\FloatTok}[1]{\textcolor[rgb]{0.25,0.63,0.44}{{#1}}}
    \newcommand{\CharTok}[1]{\textcolor[rgb]{0.25,0.44,0.63}{{#1}}}
    \newcommand{\StringTok}[1]{\textcolor[rgb]{0.25,0.44,0.63}{{#1}}}
    \newcommand{\CommentTok}[1]{\textcolor[rgb]{0.38,0.63,0.69}{\textit{{#1}}}}
    \newcommand{\OtherTok}[1]{\textcolor[rgb]{0.00,0.44,0.13}{{#1}}}
    \newcommand{\AlertTok}[1]{\textcolor[rgb]{1.00,0.00,0.00}{\textbf{{#1}}}}
    \newcommand{\FunctionTok}[1]{\textcolor[rgb]{0.02,0.16,0.49}{{#1}}}
    \newcommand{\RegionMarkerTok}[1]{{#1}}
    \newcommand{\ErrorTok}[1]{\textcolor[rgb]{1.00,0.00,0.00}{\textbf{{#1}}}}
    \newcommand{\NormalTok}[1]{{#1}}
    
    % Additional commands for more recent versions of Pandoc
    \newcommand{\ConstantTok}[1]{\textcolor[rgb]{0.53,0.00,0.00}{{#1}}}
    \newcommand{\SpecialCharTok}[1]{\textcolor[rgb]{0.25,0.44,0.63}{{#1}}}
    \newcommand{\VerbatimStringTok}[1]{\textcolor[rgb]{0.25,0.44,0.63}{{#1}}}
    \newcommand{\SpecialStringTok}[1]{\textcolor[rgb]{0.73,0.40,0.53}{{#1}}}
    \newcommand{\ImportTok}[1]{{#1}}
    \newcommand{\DocumentationTok}[1]{\textcolor[rgb]{0.73,0.13,0.13}{\textit{{#1}}}}
    \newcommand{\AnnotationTok}[1]{\textcolor[rgb]{0.38,0.63,0.69}{\textbf{\textit{{#1}}}}}
    \newcommand{\CommentVarTok}[1]{\textcolor[rgb]{0.38,0.63,0.69}{\textbf{\textit{{#1}}}}}
    \newcommand{\VariableTok}[1]{\textcolor[rgb]{0.10,0.09,0.49}{{#1}}}
    \newcommand{\ControlFlowTok}[1]{\textcolor[rgb]{0.00,0.44,0.13}{\textbf{{#1}}}}
    \newcommand{\OperatorTok}[1]{\textcolor[rgb]{0.40,0.40,0.40}{{#1}}}
    \newcommand{\BuiltInTok}[1]{{#1}}
    \newcommand{\ExtensionTok}[1]{{#1}}
    \newcommand{\PreprocessorTok}[1]{\textcolor[rgb]{0.74,0.48,0.00}{{#1}}}
    \newcommand{\AttributeTok}[1]{\textcolor[rgb]{0.49,0.56,0.16}{{#1}}}
    \newcommand{\InformationTok}[1]{\textcolor[rgb]{0.38,0.63,0.69}{\textbf{\textit{{#1}}}}}
    \newcommand{\WarningTok}[1]{\textcolor[rgb]{0.38,0.63,0.69}{\textbf{\textit{{#1}}}}}
    
    
    % Define a nice break command that doesn't care if a line doesn't already
    % exist.
    \def\br{\hspace*{\fill} \\* }
    % Math Jax compatability definitions
    \def\gt{>}
    \def\lt{<}
    % Document parameters
    \title{notebook}
    
    
    

    % Pygments definitions
    
\makeatletter
\def\PY@reset{\let\PY@it=\relax \let\PY@bf=\relax%
    \let\PY@ul=\relax \let\PY@tc=\relax%
    \let\PY@bc=\relax \let\PY@ff=\relax}
\def\PY@tok#1{\csname PY@tok@#1\endcsname}
\def\PY@toks#1+{\ifx\relax#1\empty\else%
    \PY@tok{#1}\expandafter\PY@toks\fi}
\def\PY@do#1{\PY@bc{\PY@tc{\PY@ul{%
    \PY@it{\PY@bf{\PY@ff{#1}}}}}}}
\def\PY#1#2{\PY@reset\PY@toks#1+\relax+\PY@do{#2}}

\expandafter\def\csname PY@tok@w\endcsname{\def\PY@tc##1{\textcolor[rgb]{0.73,0.73,0.73}{##1}}}
\expandafter\def\csname PY@tok@c\endcsname{\let\PY@it=\textit\def\PY@tc##1{\textcolor[rgb]{0.25,0.50,0.50}{##1}}}
\expandafter\def\csname PY@tok@cp\endcsname{\def\PY@tc##1{\textcolor[rgb]{0.74,0.48,0.00}{##1}}}
\expandafter\def\csname PY@tok@k\endcsname{\let\PY@bf=\textbf\def\PY@tc##1{\textcolor[rgb]{0.00,0.50,0.00}{##1}}}
\expandafter\def\csname PY@tok@kp\endcsname{\def\PY@tc##1{\textcolor[rgb]{0.00,0.50,0.00}{##1}}}
\expandafter\def\csname PY@tok@kt\endcsname{\def\PY@tc##1{\textcolor[rgb]{0.69,0.00,0.25}{##1}}}
\expandafter\def\csname PY@tok@o\endcsname{\def\PY@tc##1{\textcolor[rgb]{0.40,0.40,0.40}{##1}}}
\expandafter\def\csname PY@tok@ow\endcsname{\let\PY@bf=\textbf\def\PY@tc##1{\textcolor[rgb]{0.67,0.13,1.00}{##1}}}
\expandafter\def\csname PY@tok@nb\endcsname{\def\PY@tc##1{\textcolor[rgb]{0.00,0.50,0.00}{##1}}}
\expandafter\def\csname PY@tok@nf\endcsname{\def\PY@tc##1{\textcolor[rgb]{0.00,0.00,1.00}{##1}}}
\expandafter\def\csname PY@tok@nc\endcsname{\let\PY@bf=\textbf\def\PY@tc##1{\textcolor[rgb]{0.00,0.00,1.00}{##1}}}
\expandafter\def\csname PY@tok@nn\endcsname{\let\PY@bf=\textbf\def\PY@tc##1{\textcolor[rgb]{0.00,0.00,1.00}{##1}}}
\expandafter\def\csname PY@tok@ne\endcsname{\let\PY@bf=\textbf\def\PY@tc##1{\textcolor[rgb]{0.82,0.25,0.23}{##1}}}
\expandafter\def\csname PY@tok@nv\endcsname{\def\PY@tc##1{\textcolor[rgb]{0.10,0.09,0.49}{##1}}}
\expandafter\def\csname PY@tok@no\endcsname{\def\PY@tc##1{\textcolor[rgb]{0.53,0.00,0.00}{##1}}}
\expandafter\def\csname PY@tok@nl\endcsname{\def\PY@tc##1{\textcolor[rgb]{0.63,0.63,0.00}{##1}}}
\expandafter\def\csname PY@tok@ni\endcsname{\let\PY@bf=\textbf\def\PY@tc##1{\textcolor[rgb]{0.60,0.60,0.60}{##1}}}
\expandafter\def\csname PY@tok@na\endcsname{\def\PY@tc##1{\textcolor[rgb]{0.49,0.56,0.16}{##1}}}
\expandafter\def\csname PY@tok@nt\endcsname{\let\PY@bf=\textbf\def\PY@tc##1{\textcolor[rgb]{0.00,0.50,0.00}{##1}}}
\expandafter\def\csname PY@tok@nd\endcsname{\def\PY@tc##1{\textcolor[rgb]{0.67,0.13,1.00}{##1}}}
\expandafter\def\csname PY@tok@s\endcsname{\def\PY@tc##1{\textcolor[rgb]{0.73,0.13,0.13}{##1}}}
\expandafter\def\csname PY@tok@sd\endcsname{\let\PY@it=\textit\def\PY@tc##1{\textcolor[rgb]{0.73,0.13,0.13}{##1}}}
\expandafter\def\csname PY@tok@si\endcsname{\let\PY@bf=\textbf\def\PY@tc##1{\textcolor[rgb]{0.73,0.40,0.53}{##1}}}
\expandafter\def\csname PY@tok@se\endcsname{\let\PY@bf=\textbf\def\PY@tc##1{\textcolor[rgb]{0.73,0.40,0.13}{##1}}}
\expandafter\def\csname PY@tok@sr\endcsname{\def\PY@tc##1{\textcolor[rgb]{0.73,0.40,0.53}{##1}}}
\expandafter\def\csname PY@tok@ss\endcsname{\def\PY@tc##1{\textcolor[rgb]{0.10,0.09,0.49}{##1}}}
\expandafter\def\csname PY@tok@sx\endcsname{\def\PY@tc##1{\textcolor[rgb]{0.00,0.50,0.00}{##1}}}
\expandafter\def\csname PY@tok@m\endcsname{\def\PY@tc##1{\textcolor[rgb]{0.40,0.40,0.40}{##1}}}
\expandafter\def\csname PY@tok@gh\endcsname{\let\PY@bf=\textbf\def\PY@tc##1{\textcolor[rgb]{0.00,0.00,0.50}{##1}}}
\expandafter\def\csname PY@tok@gu\endcsname{\let\PY@bf=\textbf\def\PY@tc##1{\textcolor[rgb]{0.50,0.00,0.50}{##1}}}
\expandafter\def\csname PY@tok@gd\endcsname{\def\PY@tc##1{\textcolor[rgb]{0.63,0.00,0.00}{##1}}}
\expandafter\def\csname PY@tok@gi\endcsname{\def\PY@tc##1{\textcolor[rgb]{0.00,0.63,0.00}{##1}}}
\expandafter\def\csname PY@tok@gr\endcsname{\def\PY@tc##1{\textcolor[rgb]{1.00,0.00,0.00}{##1}}}
\expandafter\def\csname PY@tok@ge\endcsname{\let\PY@it=\textit}
\expandafter\def\csname PY@tok@gs\endcsname{\let\PY@bf=\textbf}
\expandafter\def\csname PY@tok@gp\endcsname{\let\PY@bf=\textbf\def\PY@tc##1{\textcolor[rgb]{0.00,0.00,0.50}{##1}}}
\expandafter\def\csname PY@tok@go\endcsname{\def\PY@tc##1{\textcolor[rgb]{0.53,0.53,0.53}{##1}}}
\expandafter\def\csname PY@tok@gt\endcsname{\def\PY@tc##1{\textcolor[rgb]{0.00,0.27,0.87}{##1}}}
\expandafter\def\csname PY@tok@err\endcsname{\def\PY@bc##1{\setlength{\fboxsep}{0pt}\fcolorbox[rgb]{1.00,0.00,0.00}{1,1,1}{\strut ##1}}}
\expandafter\def\csname PY@tok@kc\endcsname{\let\PY@bf=\textbf\def\PY@tc##1{\textcolor[rgb]{0.00,0.50,0.00}{##1}}}
\expandafter\def\csname PY@tok@kd\endcsname{\let\PY@bf=\textbf\def\PY@tc##1{\textcolor[rgb]{0.00,0.50,0.00}{##1}}}
\expandafter\def\csname PY@tok@kn\endcsname{\let\PY@bf=\textbf\def\PY@tc##1{\textcolor[rgb]{0.00,0.50,0.00}{##1}}}
\expandafter\def\csname PY@tok@kr\endcsname{\let\PY@bf=\textbf\def\PY@tc##1{\textcolor[rgb]{0.00,0.50,0.00}{##1}}}
\expandafter\def\csname PY@tok@bp\endcsname{\def\PY@tc##1{\textcolor[rgb]{0.00,0.50,0.00}{##1}}}
\expandafter\def\csname PY@tok@fm\endcsname{\def\PY@tc##1{\textcolor[rgb]{0.00,0.00,1.00}{##1}}}
\expandafter\def\csname PY@tok@vc\endcsname{\def\PY@tc##1{\textcolor[rgb]{0.10,0.09,0.49}{##1}}}
\expandafter\def\csname PY@tok@vg\endcsname{\def\PY@tc##1{\textcolor[rgb]{0.10,0.09,0.49}{##1}}}
\expandafter\def\csname PY@tok@vi\endcsname{\def\PY@tc##1{\textcolor[rgb]{0.10,0.09,0.49}{##1}}}
\expandafter\def\csname PY@tok@vm\endcsname{\def\PY@tc##1{\textcolor[rgb]{0.10,0.09,0.49}{##1}}}
\expandafter\def\csname PY@tok@sa\endcsname{\def\PY@tc##1{\textcolor[rgb]{0.73,0.13,0.13}{##1}}}
\expandafter\def\csname PY@tok@sb\endcsname{\def\PY@tc##1{\textcolor[rgb]{0.73,0.13,0.13}{##1}}}
\expandafter\def\csname PY@tok@sc\endcsname{\def\PY@tc##1{\textcolor[rgb]{0.73,0.13,0.13}{##1}}}
\expandafter\def\csname PY@tok@dl\endcsname{\def\PY@tc##1{\textcolor[rgb]{0.73,0.13,0.13}{##1}}}
\expandafter\def\csname PY@tok@s2\endcsname{\def\PY@tc##1{\textcolor[rgb]{0.73,0.13,0.13}{##1}}}
\expandafter\def\csname PY@tok@sh\endcsname{\def\PY@tc##1{\textcolor[rgb]{0.73,0.13,0.13}{##1}}}
\expandafter\def\csname PY@tok@s1\endcsname{\def\PY@tc##1{\textcolor[rgb]{0.73,0.13,0.13}{##1}}}
\expandafter\def\csname PY@tok@mb\endcsname{\def\PY@tc##1{\textcolor[rgb]{0.40,0.40,0.40}{##1}}}
\expandafter\def\csname PY@tok@mf\endcsname{\def\PY@tc##1{\textcolor[rgb]{0.40,0.40,0.40}{##1}}}
\expandafter\def\csname PY@tok@mh\endcsname{\def\PY@tc##1{\textcolor[rgb]{0.40,0.40,0.40}{##1}}}
\expandafter\def\csname PY@tok@mi\endcsname{\def\PY@tc##1{\textcolor[rgb]{0.40,0.40,0.40}{##1}}}
\expandafter\def\csname PY@tok@il\endcsname{\def\PY@tc##1{\textcolor[rgb]{0.40,0.40,0.40}{##1}}}
\expandafter\def\csname PY@tok@mo\endcsname{\def\PY@tc##1{\textcolor[rgb]{0.40,0.40,0.40}{##1}}}
\expandafter\def\csname PY@tok@ch\endcsname{\let\PY@it=\textit\def\PY@tc##1{\textcolor[rgb]{0.25,0.50,0.50}{##1}}}
\expandafter\def\csname PY@tok@cm\endcsname{\let\PY@it=\textit\def\PY@tc##1{\textcolor[rgb]{0.25,0.50,0.50}{##1}}}
\expandafter\def\csname PY@tok@cpf\endcsname{\let\PY@it=\textit\def\PY@tc##1{\textcolor[rgb]{0.25,0.50,0.50}{##1}}}
\expandafter\def\csname PY@tok@c1\endcsname{\let\PY@it=\textit\def\PY@tc##1{\textcolor[rgb]{0.25,0.50,0.50}{##1}}}
\expandafter\def\csname PY@tok@cs\endcsname{\let\PY@it=\textit\def\PY@tc##1{\textcolor[rgb]{0.25,0.50,0.50}{##1}}}

\def\PYZbs{\char`\\}
\def\PYZus{\char`\_}
\def\PYZob{\char`\{}
\def\PYZcb{\char`\}}
\def\PYZca{\char`\^}
\def\PYZam{\char`\&}
\def\PYZlt{\char`\<}
\def\PYZgt{\char`\>}
\def\PYZsh{\char`\#}
\def\PYZpc{\char`\%}
\def\PYZdl{\char`\$}
\def\PYZhy{\char`\-}
\def\PYZsq{\char`\'}
\def\PYZdq{\char`\"}
\def\PYZti{\char`\~}
% for compatibility with earlier versions
\def\PYZat{@}
\def\PYZlb{[}
\def\PYZrb{]}
\makeatother


    % Exact colors from NB
    \definecolor{incolor}{rgb}{0.0, 0.0, 0.5}
    \definecolor{outcolor}{rgb}{0.545, 0.0, 0.0}



    
    % Prevent overflowing lines due to hard-to-break entities
    \sloppy 
    % Setup hyperref package
    \hypersetup{
      breaklinks=true,  % so long urls are correctly broken across lines
      colorlinks=true,
      urlcolor=urlcolor,
      linkcolor=linkcolor,
      citecolor=citecolor,
      }
    % Slightly bigger margins than the latex defaults
    
    \geometry{verbose,tmargin=1in,bmargin=1in,lmargin=1in,rmargin=1in}
    
    

    \begin{document}
    
    
    \maketitle
    
    

    
    \subsection{1. Winter is Coming. Let's load the dataset
ASAP}\label{winter-is-coming.-lets-load-the-dataset-asap}

If you haven't heard of Game of Thrones, then you must be really good at
hiding. Game of Thrones is the hugely popular television series by HBO
based on the (also) hugely popular book series A Song of Ice and Fire by
George R.R. Martin. In this notebook, we will analyze the co-occurrence
network of the characters in the Game of Thrones books. Here, two
characters are considered to co-occur if their names appear in the
vicinity of 15 words from one another in the books.

This dataset constitutes a network and is given as a text file
describing the edges between characters, with some attributes attached
to each edge. Let's start by loading in the data for the first book A
Game of Thrones and inspect it.

    \begin{Verbatim}[commandchars=\\\{\}]
{\color{incolor}In [{\color{incolor}2}]:} \PY{c+c1}{\PYZsh{} Importing modules}
        \PY{k+kn}{import} \PY{n+nn}{pandas} \PY{k}{as} \PY{n+nn}{pd}
        
        \PY{c+c1}{\PYZsh{} Reading in datasets/book1.csv}
        \PY{n}{book1} \PY{o}{=} \PY{n}{pd}\PY{o}{.}\PY{n}{read\PYZus{}csv}\PY{p}{(}\PY{l+s+s1}{\PYZsq{}}\PY{l+s+s1}{datasets/book1.csv}\PY{l+s+s1}{\PYZsq{}}\PY{p}{)}
        
        \PY{c+c1}{\PYZsh{} Printing out the head of the dataset}
        \PY{n}{book1}\PY{o}{.}\PY{n}{head}\PY{p}{(}\PY{p}{)}
\end{Verbatim}


\begin{Verbatim}[commandchars=\\\{\}]
{\color{outcolor}Out[{\color{outcolor}2}]:}                             Source              Target        Type  weight  \textbackslash{}
        0                   Addam-Marbrand     Jaime-Lannister  Undirected       3   
        1                   Addam-Marbrand     Tywin-Lannister  Undirected       6   
        2                Aegon-I-Targaryen  Daenerys-Targaryen  Undirected       5   
        3                Aegon-I-Targaryen        Eddard-Stark  Undirected       4   
        4  Aemon-Targaryen-(Maester-Aemon)      Alliser-Thorne  Undirected       4   
        
           book  
        0     1  
        1     1  
        2     1  
        3     1  
        4     1  
\end{Verbatim}
            
    \subsection{2. Time for some Network of
Thrones}\label{time-for-some-network-of-thrones}

The resulting DataFrame book1 has 5 columns: Source, Target, Type,
weight, and book. Source and target are the two nodes that are linked by
an edge. A network can have directed or undirected edges and in this
network all the edges are undirected. The weight attribute of every edge
tells us the number of interactions that the characters have had over
the book, and the book column tells us the book number.

Once we have the data loaded as a pandas DataFrame, it's time to create
a network. We will use networkx, a network analysis library, and create
a graph object for the first book.

    \begin{Verbatim}[commandchars=\\\{\}]
{\color{incolor}In [{\color{incolor}3}]:} \PY{c+c1}{\PYZsh{} Importing modules}
        \PY{k+kn}{import} \PY{n+nn}{networkx} \PY{k}{as} \PY{n+nn}{nx}
        
        \PY{c+c1}{\PYZsh{} Creating an empty graph object}
        \PY{n}{G\PYZus{}book1} \PY{o}{=} \PY{n}{nx}\PY{o}{.}\PY{n}{Graph}\PY{p}{(}\PY{p}{)}
\end{Verbatim}


    \subsection{3. Populate the network with the
DataFrame}\label{populate-the-network-with-the-dataframe}

Currently, the graph object G\_book1 is empty. Let's now populate it
with the edges from book1. And while we're at it, let's load in the rest
of the books too!

    \begin{Verbatim}[commandchars=\\\{\}]
{\color{incolor}In [{\color{incolor}4}]:} \PY{c+c1}{\PYZsh{} Iterating through the DataFrame to add edges}
        \PY{k}{for} \PY{n}{\PYZus{}}\PY{p}{,} \PY{n}{edge} \PY{o+ow}{in} \PY{n}{book1}\PY{o}{.}\PY{n}{iterrows}\PY{p}{(}\PY{p}{)}\PY{p}{:}
            \PY{n}{G\PYZus{}book1}\PY{o}{.}\PY{n}{add\PYZus{}edge}\PY{p}{(}\PY{n}{edge}\PY{p}{[}\PY{l+s+s1}{\PYZsq{}}\PY{l+s+s1}{Source}\PY{l+s+s1}{\PYZsq{}}\PY{p}{]}\PY{p}{,} \PY{n}{edge}\PY{p}{[}\PY{l+s+s1}{\PYZsq{}}\PY{l+s+s1}{Target}\PY{l+s+s1}{\PYZsq{}}\PY{p}{]}\PY{p}{,} \PY{n}{weight}\PY{o}{=}\PY{n}{edge}\PY{p}{[}\PY{l+s+s1}{\PYZsq{}}\PY{l+s+s1}{weight}\PY{l+s+s1}{\PYZsq{}}\PY{p}{]}\PY{p}{)}
        
            \PY{c+c1}{\PYZsh{} Creating a list of networks for all the books}
        \PY{n}{books} \PY{o}{=} \PY{p}{[}\PY{n}{G\PYZus{}book1}\PY{p}{]}
        \PY{n}{book\PYZus{}fnames} \PY{o}{=} \PY{p}{[}\PY{l+s+s1}{\PYZsq{}}\PY{l+s+s1}{datasets/book2.csv}\PY{l+s+s1}{\PYZsq{}}\PY{p}{,} \PY{l+s+s1}{\PYZsq{}}\PY{l+s+s1}{datasets/book3.csv}\PY{l+s+s1}{\PYZsq{}}\PY{p}{,} \PY{l+s+s1}{\PYZsq{}}\PY{l+s+s1}{datasets/book4.csv}\PY{l+s+s1}{\PYZsq{}}\PY{p}{,} \PY{l+s+s1}{\PYZsq{}}\PY{l+s+s1}{datasets/book5.csv}\PY{l+s+s1}{\PYZsq{}}\PY{p}{]}
        \PY{k}{for} \PY{n}{book\PYZus{}fname} \PY{o+ow}{in} \PY{n}{book\PYZus{}fnames}\PY{p}{:}
            \PY{n}{book} \PY{o}{=} \PY{n}{pd}\PY{o}{.}\PY{n}{read\PYZus{}csv}\PY{p}{(}\PY{n}{book\PYZus{}fname}\PY{p}{)}
            \PY{n}{G\PYZus{}book} \PY{o}{=} \PY{n}{nx}\PY{o}{.}\PY{n}{Graph}\PY{p}{(}\PY{p}{)}
            \PY{k}{for} \PY{n}{\PYZus{}}\PY{p}{,} \PY{n}{edge} \PY{o+ow}{in} \PY{n}{book}\PY{o}{.}\PY{n}{iterrows}\PY{p}{(}\PY{p}{)}\PY{p}{:}
                \PY{n}{G\PYZus{}book}\PY{o}{.}\PY{n}{add\PYZus{}edge}\PY{p}{(}\PY{n}{edge}\PY{p}{[}\PY{l+s+s1}{\PYZsq{}}\PY{l+s+s1}{Source}\PY{l+s+s1}{\PYZsq{}}\PY{p}{]}\PY{p}{,} \PY{n}{edge}\PY{p}{[}\PY{l+s+s1}{\PYZsq{}}\PY{l+s+s1}{Target}\PY{l+s+s1}{\PYZsq{}}\PY{p}{]}\PY{p}{,} \PY{n}{weight}\PY{o}{=}\PY{n}{edge}\PY{p}{[}\PY{l+s+s1}{\PYZsq{}}\PY{l+s+s1}{weight}\PY{l+s+s1}{\PYZsq{}}\PY{p}{]}\PY{p}{)}
            \PY{n}{books}\PY{o}{.}\PY{n}{append}\PY{p}{(}\PY{n}{G\PYZus{}book}\PY{p}{)}
\end{Verbatim}


    \subsection{4. Finding the most important character in Game of
Thrones}\label{finding-the-most-important-character-in-game-of-thrones}

Is it Jon Snow, Tyrion, Daenerys, or someone else? Let's see! Network
Science offers us many different metrics to measure the importance of a
node in a network. Note that there is no "correct" way of calculating
the most important node in a network, every metric has a different
meaning.

First, let's measure the importance of a node in a network by looking at
the number of neighbors it has, that is, the number of nodes it is
connected to. For example, an influential account on Twitter, where the
follower-followee relationship forms the network, is an account which
has a high number of followers. This measure of importance is called
degree centrality.

Using this measure, let's extract the top ten important characters from
the first book (book{[}0{]}) and the fifth book (book{[}4{]}).

    \begin{Verbatim}[commandchars=\\\{\}]
{\color{incolor}In [{\color{incolor}5}]:} \PY{c+c1}{\PYZsh{} Calculating the degree centrality of book 1}
        \PY{n}{deg\PYZus{}cen\PYZus{}book1} \PY{o}{=} \PY{n}{nx}\PY{o}{.}\PY{n}{degree\PYZus{}centrality}\PY{p}{(}\PY{n}{books}\PY{p}{[}\PY{l+m+mi}{0}\PY{p}{]}\PY{p}{)}
        
        \PY{c+c1}{\PYZsh{} Calculating the degree centrality of book 5}
        \PY{n}{deg\PYZus{}cen\PYZus{}book5} \PY{o}{=} \PY{n}{nx}\PY{o}{.}\PY{n}{degree\PYZus{}centrality}\PY{p}{(}\PY{n}{books}\PY{p}{[}\PY{l+m+mi}{4}\PY{p}{]}\PY{p}{)}
        
        \PY{c+c1}{\PYZsh{} Sorting the dictionaries according to their degree centrality and storing the top 10}
        \PY{n}{sorted\PYZus{}deg\PYZus{}cen\PYZus{}book1} \PY{o}{=} \PY{n+nb}{sorted}\PY{p}{(}\PY{n}{deg\PYZus{}cen\PYZus{}book1}\PY{o}{.}\PY{n}{items}\PY{p}{(}\PY{p}{)}\PY{p}{,} \PY{n}{key}\PY{o}{=}\PY{k}{lambda} \PY{n}{x}\PY{p}{:}\PY{n}{x}\PY{p}{[}\PY{l+m+mi}{1}\PY{p}{]}\PY{p}{,} \PY{n}{reverse}\PY{o}{=}\PY{k+kc}{True}\PY{p}{)}
        
        \PY{c+c1}{\PYZsh{} Sorting the dictionaries according to their degree centrality and storing the top 10}
        \PY{n}{sorted\PYZus{}deg\PYZus{}cen\PYZus{}book5} \PY{o}{=} \PY{n+nb}{sorted}\PY{p}{(}\PY{n}{deg\PYZus{}cen\PYZus{}book5}\PY{o}{.}\PY{n}{items}\PY{p}{(}\PY{p}{)}\PY{p}{,} \PY{n}{key}\PY{o}{=}\PY{k}{lambda} \PY{n}{x}\PY{p}{:}\PY{n}{x}\PY{p}{[}\PY{l+m+mi}{1}\PY{p}{]}\PY{p}{,} \PY{n}{reverse}\PY{o}{=}\PY{k+kc}{True}\PY{p}{)}
        
        \PY{c+c1}{\PYZsh{} Printing out the top 10 of book1 and book5}
        \PY{k}{def} \PY{n+nf}{print\PYZus{}name}\PY{p}{(}\PY{n}{data}\PY{p}{)}\PY{p}{:}
            \PY{n}{df} \PY{o}{=} \PY{n}{pd}\PY{o}{.}\PY{n}{DataFrame}\PY{p}{(}\PY{n}{data}\PY{p}{)}
            \PY{n}{name} \PY{o}{=} \PY{p}{[}\PY{n}{x}\PY{p}{[}\PY{l+m+mi}{0}\PY{p}{]} \PY{k}{for} \PY{n}{\PYZus{}}\PY{p}{,}\PY{n}{x} \PY{o+ow}{in} \PY{n}{df}\PY{o}{.}\PY{n}{iterrows}\PY{p}{(}\PY{p}{)}\PY{p}{]}
            \PY{n+nb}{print}\PY{p}{(}\PY{n}{name}\PY{p}{[}\PY{l+m+mi}{0}\PY{p}{:}\PY{l+m+mi}{10}\PY{p}{]}\PY{p}{)}
        \PY{n}{print\PYZus{}name}\PY{p}{(}\PY{n}{sorted\PYZus{}deg\PYZus{}cen\PYZus{}book1}\PY{p}{)}
        \PY{n}{print\PYZus{}name}\PY{p}{(}\PY{n}{sorted\PYZus{}deg\PYZus{}cen\PYZus{}book5}\PY{p}{)}
\end{Verbatim}


    \begin{Verbatim}[commandchars=\\\{\}]
['Eddard-Stark', 'Robert-Baratheon', 'Tyrion-Lannister', 'Catelyn-Stark', 'Jon-Snow', 'Robb-Stark', 'Sansa-Stark', 'Bran-Stark', 'Cersei-Lannister', 'Joffrey-Baratheon']
['Jon-Snow', 'Daenerys-Targaryen', 'Stannis-Baratheon', 'Tyrion-Lannister', 'Theon-Greyjoy', 'Cersei-Lannister', 'Barristan-Selmy', 'Hizdahr-zo-Loraq', 'Asha-Greyjoy', 'Melisandre']

    \end{Verbatim}

    \subsection{5. Evolution of importance of characters over the
books}\label{evolution-of-importance-of-characters-over-the-books}

According to degree centrality, the most important character in the
first book is Eddard Stark but he is not even in the top 10 of the fifth
book. The importance of characters changes over the course of five books
because, you know, stuff happens... ;)

Let's look at the evolution of degree centrality of a couple of
characters like Eddard Stark, Jon Snow, and Tyrion, which showed up in
the top 10 of degree centrality in the first book.

    \begin{Verbatim}[commandchars=\\\{\}]
{\color{incolor}In [{\color{incolor}7}]:} \PY{o}{\PYZpc{}}\PY{k}{matplotlib} inline
        
        \PY{c+c1}{\PYZsh{} Creating a list of degree centrality of all the books}
        \PY{n}{evol} \PY{o}{=} \PY{p}{[}\PY{n}{nx}\PY{o}{.}\PY{n}{degree\PYZus{}centrality}\PY{p}{(}\PY{n}{book}\PY{p}{)} \PY{k}{for} \PY{n}{book} \PY{o+ow}{in} \PY{n}{books}\PY{p}{]}
         
        \PY{c+c1}{\PYZsh{} Creating a DataFrame from the list of degree centralities in all the books}
        \PY{n}{degree\PYZus{}evol\PYZus{}df} \PY{o}{=} \PY{n}{pd}\PY{o}{.}\PY{n}{DataFrame}\PY{o}{.}\PY{n}{from\PYZus{}records}\PY{p}{(}\PY{n}{evol}\PY{p}{)}
        \PY{c+c1}{\PYZsh{} Plotting the degree centrality evolution of Eddard\PYZhy{}Stark, Tyrion\PYZhy{}Lannister and Jon\PYZhy{}Snow}
        \PY{n}{degree\PYZus{}evol\PYZus{}df}\PY{p}{[}\PY{p}{[}\PY{l+s+s1}{\PYZsq{}}\PY{l+s+s1}{Eddard\PYZhy{}Stark}\PY{l+s+s1}{\PYZsq{}}\PY{p}{,} \PY{l+s+s1}{\PYZsq{}}\PY{l+s+s1}{Tyrion\PYZhy{}Lannister}\PY{l+s+s1}{\PYZsq{}}\PY{p}{,} \PY{l+s+s1}{\PYZsq{}}\PY{l+s+s1}{Jon\PYZhy{}Snow}\PY{l+s+s1}{\PYZsq{}}\PY{p}{]}\PY{p}{]}\PY{o}{.}\PY{n}{plot}\PY{p}{(}\PY{p}{)}
\end{Verbatim}


\begin{Verbatim}[commandchars=\\\{\}]
{\color{outcolor}Out[{\color{outcolor}7}]:} <matplotlib.axes.\_subplots.AxesSubplot at 0x1a5207c7dd8>
\end{Verbatim}
            
    \begin{center}
    \adjustimage{max size={0.9\linewidth}{0.9\paperheight}}{output_9_1.png}
    \end{center}
    { \hspace*{\fill} \\}
    
    \subsection{6. What's up with Stannis
Baratheon?}\label{whats-up-with-stannis-baratheon}

We can see that the importance of Eddard Stark dies off as the book
series progresses. With Jon Snow, there is a drop in the fourth book but
a sudden rise in the fifth book.

Now let's look at various other measures like betweenness centrality and
PageRank to find important characters in our Game of Thrones character
co-occurrence network and see if we can uncover some more interesting
facts about this network. Let's plot the evolution of betweenness
centrality of this network over the five books. We will take the
evolution of the top four characters of every book and plot it.

    \begin{Verbatim}[commandchars=\\\{\}]
{\color{incolor}In [{\color{incolor}9}]:} \PY{c+c1}{\PYZsh{} Creating a list of betweenness centrality of all the books just like we did for degree centrality}
        \PY{n}{evol} \PY{o}{=} \PY{p}{[}\PY{n}{nx}\PY{o}{.}\PY{n}{betweenness\PYZus{}centrality}\PY{p}{(}\PY{n}{book}\PY{p}{,} \PY{n}{weight}\PY{o}{=}\PY{l+s+s1}{\PYZsq{}}\PY{l+s+s1}{weight}\PY{l+s+s1}{\PYZsq{}}\PY{p}{)} \PY{k}{for} \PY{n}{book} \PY{o+ow}{in} \PY{n}{books}\PY{p}{]}
        
        \PY{c+c1}{\PYZsh{} Making a DataFrame from the list}
        \PY{n}{betweenness\PYZus{}evol\PYZus{}df} \PY{o}{=} \PY{n}{pd}\PY{o}{.}\PY{n}{DataFrame}\PY{o}{.}\PY{n}{from\PYZus{}records}\PY{p}{(}\PY{n}{evol}\PY{p}{)}\PY{o}{.}\PY{n}{fillna}\PY{p}{(}\PY{l+m+mi}{0}\PY{p}{)}
        
        \PY{c+c1}{\PYZsh{} Finding the top 4 characters in every book}
        \PY{n}{set\PYZus{}of\PYZus{}char} \PY{o}{=} \PY{n+nb}{set}\PY{p}{(}\PY{p}{)}
        \PY{k}{for} \PY{n}{i} \PY{o+ow}{in} \PY{n+nb}{range}\PY{p}{(}\PY{l+m+mi}{5}\PY{p}{)}\PY{p}{:}
            \PY{n}{set\PYZus{}of\PYZus{}char} \PY{o}{|}\PY{o}{=} \PY{n+nb}{set}\PY{p}{(}\PY{n+nb}{list}\PY{p}{(}\PY{n}{betweenness\PYZus{}evol\PYZus{}df}\PY{o}{.}\PY{n}{T}\PY{p}{[}\PY{n}{i}\PY{p}{]}\PY{o}{.}\PY{n}{sort\PYZus{}values}\PY{p}{(}\PY{n}{ascending}\PY{o}{=}\PY{k+kc}{False}\PY{p}{)}\PY{p}{[}\PY{l+m+mi}{0}\PY{p}{:}\PY{l+m+mi}{4}\PY{p}{]}\PY{o}{.}\PY{n}{index}\PY{p}{)}\PY{p}{)}
        \PY{n}{list\PYZus{}of\PYZus{}char} \PY{o}{=} \PY{n+nb}{list}\PY{p}{(}\PY{n}{set\PYZus{}of\PYZus{}char}\PY{p}{)}
        
        \PY{c+c1}{\PYZsh{} Plotting the evolution of the top characters}
        \PY{n}{betweenness\PYZus{}evol\PYZus{}df}\PY{p}{[}\PY{n}{list\PYZus{}of\PYZus{}char}\PY{p}{]}\PY{o}{.}\PY{n}{plot}\PY{p}{(}\PY{n}{figsize}\PY{o}{=}\PY{p}{(}\PY{l+m+mi}{13}\PY{p}{,}\PY{l+m+mi}{7}\PY{p}{)}\PY{p}{)}
\end{Verbatim}


\begin{Verbatim}[commandchars=\\\{\}]
{\color{outcolor}Out[{\color{outcolor}9}]:} <matplotlib.axes.\_subplots.AxesSubplot at 0x1a5207d0898>
\end{Verbatim}
            
    \begin{center}
    \adjustimage{max size={0.9\linewidth}{0.9\paperheight}}{output_11_1.png}
    \end{center}
    { \hspace*{\fill} \\}
    
    \subsection{7. What does the Google PageRank algorithm tell us about
Game of
Thrones?}\label{what-does-the-google-pagerank-algorithm-tell-us-about-game-of-thrones}

We see a peculiar rise in the importance of Stannis Baratheon over the
books. In the fifth book, he is significantly more important than other
characters in the network, even though he is the third most important
character according to degree centrality.

PageRank was the initial way Google ranked web pages. It evaluates the
inlinks and outlinks of webpages in the world wide web, which is,
essentially, a directed network. Let's look at the importance of
characters in the Game of Thrones network according to PageRank.

    \begin{Verbatim}[commandchars=\\\{\}]
{\color{incolor}In [{\color{incolor}31}]:} \PY{c+c1}{\PYZsh{} Creating a list of pagerank of all the characters in all the books}
         \PY{n}{evol} \PY{o}{=} \PY{p}{[}\PY{n}{nx}\PY{o}{.}\PY{n}{pagerank}\PY{p}{(}\PY{n}{book}\PY{p}{)} \PY{k}{for} \PY{n}{book} \PY{o+ow}{in} \PY{n}{books}\PY{p}{]}
         
         \PY{c+c1}{\PYZsh{} Making a DataFrame from the list}
         \PY{n}{pagerank\PYZus{}evol\PYZus{}df} \PY{o}{=} \PY{n}{pd}\PY{o}{.}\PY{n}{DataFrame}\PY{o}{.}\PY{n}{from\PYZus{}records}\PY{p}{(}\PY{n}{evol}\PY{p}{)}
         
         \PY{c+c1}{\PYZsh{} Finding the top 4 characters in every book}
         \PY{n}{set\PYZus{}of\PYZus{}char} \PY{o}{=} \PY{n+nb}{set}\PY{p}{(}\PY{p}{)}
         \PY{k}{for} \PY{n}{i} \PY{o+ow}{in} \PY{n+nb}{range}\PY{p}{(}\PY{l+m+mi}{5}\PY{p}{)}\PY{p}{:}
             \PY{n}{set\PYZus{}of\PYZus{}char} \PY{o}{|}\PY{o}{=} \PY{n+nb}{set}\PY{p}{(}\PY{n+nb}{list}\PY{p}{(}\PY{n}{pagerank\PYZus{}evol\PYZus{}df}\PY{o}{.}\PY{n}{T}\PY{p}{[}\PY{n}{i}\PY{p}{]}\PY{o}{.}\PY{n}{sort\PYZus{}values}\PY{p}{(}\PY{n}{ascending}\PY{o}{=}\PY{k+kc}{False}\PY{p}{)}\PY{p}{[}\PY{l+m+mi}{0}\PY{p}{:}\PY{l+m+mi}{4}\PY{p}{]}\PY{o}{.}\PY{n}{index}\PY{p}{)}\PY{p}{)}
         \PY{n}{list\PYZus{}of\PYZus{}char} \PY{o}{=} \PY{n+nb}{list}\PY{p}{(}\PY{n}{set\PYZus{}of\PYZus{}char}\PY{p}{)}
         
         \PY{c+c1}{\PYZsh{} Plotting the top characters}
         \PY{n}{pagerank\PYZus{}evol\PYZus{}df}\PY{p}{[}\PY{n}{list\PYZus{}of\PYZus{}char}\PY{p}{]}\PY{o}{.}\PY{n}{plot}\PY{p}{(}\PY{n}{figsize}\PY{o}{=}\PY{p}{(}\PY{l+m+mi}{13}\PY{p}{,}\PY{l+m+mi}{7}\PY{p}{)}\PY{p}{)}
\end{Verbatim}


\begin{Verbatim}[commandchars=\\\{\}]
{\color{outcolor}Out[{\color{outcolor}31}]:} <matplotlib.axes.\_subplots.AxesSubplot at 0x1a5223815c0>
\end{Verbatim}
            
    \begin{center}
    \adjustimage{max size={0.9\linewidth}{0.9\paperheight}}{output_13_1.png}
    \end{center}
    { \hspace*{\fill} \\}
    
    \subsection{8. Correlation between different
measures}\label{correlation-between-different-measures}

Stannis, Jon Snow, and Daenerys are the most important characters in the
fifth book according to PageRank. Eddard Stark follows a similar curve
but for degree centrality and betweenness centrality: He is important in
the first book but dies into oblivion over the book series.

We have seen three different measures to calculate the importance of a
node in a network, and all of them tells us something about the
characters and their importance in the co-occurrence network. We see
some names pop up in all three measures so maybe there is a strong
correlation between them?

Let's look at the correlation between PageRank, betweenness centrality
and degree centrality for the fifth book using Pearson correlation.

    \begin{Verbatim}[commandchars=\\\{\}]
{\color{incolor}In [{\color{incolor}32}]:} \PY{c+c1}{\PYZsh{} Creating a list of pagerank, betweenness centrality, degree centrality}
         \PY{c+c1}{\PYZsh{} of all the characters in the fifth book.}
         \PY{n}{measures} \PY{o}{=} \PY{p}{[}\PY{n}{nx}\PY{o}{.}\PY{n}{pagerank}\PY{p}{(}\PY{n}{books}\PY{p}{[}\PY{l+m+mi}{4}\PY{p}{]}\PY{p}{)}\PY{p}{,} 
                     \PY{n}{nx}\PY{o}{.}\PY{n}{betweenness\PYZus{}centrality}\PY{p}{(}\PY{n}{books}\PY{p}{[}\PY{l+m+mi}{4}\PY{p}{]}\PY{p}{,} \PY{n}{weight}\PY{o}{=}\PY{l+s+s1}{\PYZsq{}}\PY{l+s+s1}{weight}\PY{l+s+s1}{\PYZsq{}}\PY{p}{)}\PY{p}{,} 
                     \PY{n}{nx}\PY{o}{.}\PY{n}{degree\PYZus{}centrality}\PY{p}{(}\PY{n}{books}\PY{p}{[}\PY{l+m+mi}{4}\PY{p}{]}\PY{p}{)}\PY{p}{]}
         
         \PY{c+c1}{\PYZsh{} Creating the correlation DataFrame}
         \PY{n}{cor} \PY{o}{=} \PY{n}{pd}\PY{o}{.}\PY{n}{DataFrame}\PY{o}{.}\PY{n}{from\PYZus{}records}\PY{p}{(}\PY{n}{measures}\PY{p}{)}
         
         \PY{c+c1}{\PYZsh{} Calculating the correlation}
         \PY{n}{cor}\PY{o}{.}\PY{n}{T}\PY{o}{.}\PY{n}{corr}\PY{p}{(}\PY{p}{)}
\end{Verbatim}


\begin{Verbatim}[commandchars=\\\{\}]
{\color{outcolor}Out[{\color{outcolor}32}]:}           0         1         2
         0  1.000000  0.793372  0.971493
         1  0.793372  1.000000  0.833816
         2  0.971493  0.833816  1.000000
\end{Verbatim}
            
    \subsection{9. Conclusion}\label{conclusion}

We see a high correlation between these three measures for our character
co-occurrence network.

So we've been looking at different ways to find the important characters
in the Game of Thrones co-occurrence network. According to degree
centrality, Eddard Stark is the most important character initially in
the books. But who is/are the most important character(s) in the fifth
book according to these three measures?

    \begin{Verbatim}[commandchars=\\\{\}]
{\color{incolor}In [{\color{incolor}33}]:} \PY{c+c1}{\PYZsh{} Finding the most important character in the fifth book,  }
         \PY{c+c1}{\PYZsh{} according to degree centrality, betweenness centrality and pagerank.}
         \PY{n}{p\PYZus{}rank}\PY{p}{,} \PY{n}{b\PYZus{}cent}\PY{p}{,} \PY{n}{d\PYZus{}cent} \PY{o}{=} \PY{n}{cor}\PY{o}{.}\PY{n}{idxmax}\PY{p}{(}\PY{n}{axis}\PY{o}{=}\PY{l+m+mi}{1}\PY{p}{)}
         
         \PY{c+c1}{\PYZsh{} Printing out the top character accoding to the three measures}
         \PY{n+nb}{print}\PY{p}{(}\PY{n}{p\PYZus{}rank}\PY{p}{,} \PY{n}{b\PYZus{}cent}\PY{p}{,} \PY{n}{d\PYZus{}cent}\PY{p}{)}
\end{Verbatim}


    \begin{Verbatim}[commandchars=\\\{\}]
Jon-Snow Stannis-Baratheon Jon-Snow

    \end{Verbatim}


    % Add a bibliography block to the postdoc
    
    
    
    \end{document}
